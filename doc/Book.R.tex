%%%%%%%%%%%%%%%%%%%%%%%%%%%%%%%%%%%%%%%%%%%%%%%%%%%%%%%%%%%%%%%%%%%%%%%%%%%%%%%
% Introducción a las estadísticas con R
%%%%%%%%%%%%%%%%%%%%%%%%%%%%%%%%%%%%%%%%%%%%%%%%%%%%%%%%%%%%%%%%%%%%%%%%%%%%%%%

\documentclass[spanish]{extbook}
\usepackage[T1]{fontenc}
\usepackage[utf8]{inputenc}
%\usepackage[a4paper,showframe]{geometry}
\usepackage[a4paper]{geometry}
\usepackage{color}
\usepackage{amsmath}
\usepackage{graphicx}
\usepackage{listings}
\usepackage{float}
\usepackage[font=small,labelfont=bf]{caption}
\usepackage{graphicx}
\usepackage{parskip}
\usepackage[hidelinks]{hyperref}
\usepackage[capitalise]{cleveref}
\usepackage{bookmark}
\usepackage[spanish]{babel}
\usepackage{times}
\usepackage{color}
\usepackage[spanish]{babel}
\geometry{marginparwidth=0.5cm,a4paper,verbose,tmargin=2cm,bmargin=2cm,lmargin=3cm,rmargin=1cm,headheight=1cm,headsep=1cm,footskip=1cm}
\usepackage[strict]{changepage}

\usepackage{framed}

% environment derived from framed.sty: see leftbar environment definition
\definecolor{formalshade}{rgb}{0.95,0.95,1}
\definecolor{darkblue}{rgb}{0.0, 0.0, 0.55}

\newenvironment{tradnote}{%
  \def\FrameCommand{%
    \hspace{1pt}%
    {\color{darkblue}\vrule width 2pt}%
    {\color{formalshade}\vrule width 4pt}%
    \colorbox{formalshade}%
  }%
  \vspace{12pt}
  \MakeFramed{\advance\hsize-\width\FrameRestore}%
  \noindent\hspace{-4.55pt}% disable indenting first paragraph
  \begin{adjustwidth}{}{7pt}%
  \vspace{1pt}%
  \textbf{Nota del Traductor:\\}%
}
{%
  \vspace{6pt}\end{adjustwidth}\endMakeFramed%
}


\definecolor{gray97}{gray}{.97}
\definecolor{gray75}{gray}{.75}
\definecolor{gray45}{gray}{.45}

\crefdefaultlabelformat{\textbf{#2#1#3}}
\crefname{figure}{\textbf{Fig.}}{\textbf{Figures}}

\geometry{verbose}
\makeatletter
\numberwithin{equation}{section}
\numberwithin{figure}{section}
\makeatother
\addto\shorthandsspanish{\spanishdeactivate{~<>}}
\renewcommand{\lstlistingname}{Listado de código}
\setlength{\parindent}{0pt}
\renewcommand{\baselinestretch}{0.96}

\usepackage{listings}
\usepackage{color}

%\lstset{ frame=Ltb,
%     framerule=0pt,
%     aboveskip=0.5cm,
%     framextopmargin=3pt,
%     framexbottommargin=3pt,
%     framexleftmargin=0.4cm,
%     framesep=0pt,
%     rulesep=.4pt,
%     backgroundcolor=\color{gray97},
%     rulesepcolor=\color{black},
%     %
%     stringstyle=\ttfamily,
%     showstringspaces = false,
%     basicstyle=\small\ttfamily,
%     commentstyle=\color{gray45},
%     keywordstyle=\bfseries,
%     %
%     numbers=left,
%     numbersep=15pt,
%     numberstyle=\tiny,
%     numberfirstline = false,
%     breaklines=true,
%   }

% minimizar fragmentado de listados
\lstnewenvironment{listing}[1][]
   {\lstset{#1}\pagebreak[0]}{\pagebreak[0]}

\lstdefinestyle{consola}
   {basicstyle=\scriptsize\bf\ttfamily,
    backgroundcolor=\color{gray75},
   }

\lstdefinestyle{R}
{
	language=R,
}

\definecolor{dkgreen}{rgb}{0,0.6,0}
\definecolor{gray}{rgb}{0.5,0.5,0.5}
\definecolor{mauve}{rgb}{0.58,0,0.82}

%%%%%%%%%%%%%%%%%%%%%%%%%%%%%%%%%%%%%%%%%%%%%%%%%%%%%%%%%%%%%%%%%%%%%%%%%%%%%%%
% Estilo del código
%%%%%%%%%%%%%%%%%%%%%%%%%%%%%%%%%%%%%%%%%%%%%%%%%%%%%%%%%%%%%%%%%%%%%%%%%%%%%%%
\lstset{frame=tb,
  framerule=0pt,
  language=R,
  aboveskip=3mm,
  belowskip=3mm,
  showstringspaces=false,
  columns=flexible,
  basicstyle={\small\ttfamily},
  numbers=none,
  numberstyle=\tiny\color{gray},
  keywordstyle=\color{blue},
  commentstyle=\color{dkgreen},
  stringstyle=\color{mauve},
  breaklines=true,
  breakatwhitespace=true,
  tabsize=4
}

%%%%%%%%%%%%%%%%%%%%%%%%%%%%%%%%%%%%%%%%%%%%%%%%%%%%%%%%%%%%%%%%%%%%%%%%%%%%%%%
% Cuerpo principal
%%%%%%%%%%%%%%%%%%%%%%%%%%%%%%%%%%%%%%%%%%%%%%%%%%%%%%%%%%%%%%%%%%%%%%%%%%%%%%%
\title{Introducción a las estadísticas con R}
\date{2018-01-20}
\author{Peter Daalgard\\
    \small{Traducido al español por Patricio Moracho}
}
\begin{document}

\maketitle
%%%%%%%%%%%%%%%%%%%%%%%%%%%%%%%%%%%%%%%%%%%%%%%%%%%%%%%%%%%%%%%%%%%%%%%%%%%%%%%
%% Prefacio
%%%%%%%%%%%%%%%%%%%%%%%%%%%%%%%%%%%%%%%%%%%%%%%%%%%%%%%%%%%%%%%%%%%%%%%%%%%%%%%
\chapter*{Prefacio}

\textbf{R} es un programa informático para estadístas, disponible a través de
Internet bajo la Licencia Pública General (\textbf{GPL}). Es decir, se
suministra con una licencia que nos permite utilizarlo libremente, distribuirlo
o incluso venderlo, siempre que el receptor tenga los mismos derechos y el
código fuente esté disponible libremente. Existen versiones para Microsoft
Windows XP o posteriores, para una variedad de plataformas Unix y Linux, y para
Apple Macintosh OS X. 

\textbf{R} proporciona un entorno en el que puede realizar análisis
estadísticos y producir gráficos.  En realidad es un lenguaje de programación
completo, aunque eso, en este libro sólo se describe marginalmente. Aquí nos
conformamos con aprender los conceptos elementales y ver varios ejemplos. 

\textbf{R} está diseñado de tal manera que siempre es posible realizar más
cálculos sobre los resultados de un procedimiento estadístico. Además, la
funcionalidad para la presentación gráfica de los datos permite tanto métodos
simples, como por ejemplo \texttt{plot(x, y)}, como también la posibilidad de
un control mucho más fino del aspecto de las gráficas. 

El hecho que \textbf{R} esté basado en un lenguaje formal de computadora le da
una tremenda flexibilidad.  Otros sistemas presentan interfaces más sencillas
en términos de menús y formularios, pero a menudo la aparente facilidad de uso
se convierte en un obstáculo a largo plazo. Aunque las estadísticas elementales
se presentan a menudo como una colección de procedimientos fijos, el análisis
de datos moderadamente complejos requiere la creación de modelos estadísticos
ad-hoc, lo que hace que la flexibilidad añadida de \textbf{R} sea altamente
deseable.

\textbf{R} debe su nombre a una típica humorada de Internet. Es posible que
haya oído hablar del lenguaje de programación C (cuyo nombre es otra historia
en sí misma). Inspirados por esto, Becker y Chambers eligieron, a principios de
los años ochenta, llamar a su nuevo lenguaje de programación estadística S.
Este lenguaje se desarrolló aún más con el producto comercial S-PLUS, que a
finales de la década era de uso generalizado entre los estadísticos de todo
tipo. Ross Ihaka y Robert Gentleman de la Universidad de Auckland, Nueva
Zelanda, eligieron escribir una versión reducida de S para propósitos de
enseñanza, y ¿qué era más natural que elegir la letra inmediatamente anterior?
Las iniciales de Ross y Robert también pueden haber jugado un papel importante
en la elección del nombre.

En 1995, Martin Maechler persuadió a Ross y Robert para que publicaran el
código fuente de \textbf{R} bajo la licencia \textbf{GPL}. Esto coincidió con
el auge del software de código abierto impulsado por el sistema Linux.
\textbf{R} pronto logró llenar un hueco en gente como yo que tenía la intención
de usar Linux para computación estadística, pero no tenía ningún paquete
estadístico disponible en ese momento. Se creó una lista de correo para la
comunicación de informes de fallos y discusiones sobre el desarrollo de
\textbf{R}.

En agosto de 1997, me invitaron a unirme a un equipo internacional extendido
cuyos miembros colaboran a través de Internet y que ha controlado el desarrollo
de \textbf{R} desde entonces. Posteriormente, el equipo básico se amplió varias
veces y actualmente cuenta con 19 miembros. El 29 de febrero de 2000, se
publicó la versión 1.0.0.0. Al momento de este escrito, la versión actual es la
2.6.2. Este libro se basó originalmente en una serie de notas elaboradas para
el curso de Estadística Básica para Investigadores en Salud de la Facultad de
Ciencias de la Salud de la Universidad de Copenhague. El curso tenía como
objetivo principal, los estudiantes del doctorado en medicina. Sin embargo, el
material se ha revisado sustancialmente y espero que sea útil para un público
más amplio, aunque sigue habiendo algunos sesgos bioestadísticos, especialmente
en la elección de ejemplos. En los últimos años, el curso de Práctica
Estadística en Epidemiología, que se ha celebrado anualmente en Tartu
(Estonia), ha sido una fuente importante de inspiración y experiencia en la
introducción de jóvenes estadísticos y epidemiólogos a \textbf{R}.  Este libro
no es un manual para \textbf{R}.  La idea es introducir una serie de conceptos
y técnicas básicas que deberían permitir al lector comenzar con estadísticas
prácticas. En cuanto a los métodos prácticos, el libro cubre una currícula
razonable para los estudiantes de primer año de estadística teórica, así como
para los estudiantes de ingeniería. Estos grupos eventualmente necesitarán ir
más allá y estudiar modelos más complejos, así como técnicas generales que
involucren programación real en el lenguaje \textbf{R}.

Para los áreas en las que las estadísticas elementales se enseñan
principalmente como una herramienta, el libro va un poco más allá de lo que se
enseña comúnmente a nivel universitario. Los métodos de regresión múltiple o el
análisis de experimentos multifactoriales rara vez se enseñan a ese nivel, pero
pueden convertirse rápidamente en esenciales para la investigación práctica. He
recopilado los métodos más simples al principio para hacer el libro más legible
también en un nivel elemental. Sin embargo, para mantener el material técnico
consistente, los capítulos 1 y 2 sí incluyen material que algunos lectores
querrán omitir. Por lo tanto, el libro pretende ser útil para varios grupos,
pero no pretenderé que pueda valerse por sí solo para ninguno de ellos. He
incluido breves secciones teóricas en relación con los diversos métodos, pero
más que como material didáctico, éstos deberían servir de recordatorios o
quizás como aperitivos para los lectores que son nuevos en el mundo de la
estadística.


\cleardoublepage
\pdfbookmark{\contentsname}{Contenido}

\tableofcontents
\include{01.Capitulo.1}
%%%%%%%%%%%%%%%%%%%%%%%%%%%%%%%%%%%%%%%%%%%%%%%%%%%%%%%%%%%%%%%%%%%%%%%%%%%%%%%
%% El Entorno de R
%%%%%%%%%%%%%%%%%%%%%%%%%%%%%%%%%%%%%%%%%%%%%%%%%%%%%%%%%%%%%%%%%%%%%%%%%%%%%%%
\chapter{El entorno de R}

Este capítulo comprende algunos aspectos prácticos del trabajo con \textbf{R}.
Describe cuestiones relacionadas con la estructura del espacio de trabajo, los
dispositivos gráficos y sus parámetros, y la programación elemental, e incluye
una discusión bastante extensa, aunque lejos de ser completa, sobre el ingreso
de datos.

%%%%%%%%%%%%%%%%%%%%%%%%%%%%%%%%%%%%%%%%%%%%%%%%%%%%%%%%%%%%%%%%%%%%%%%%%%%%%%%
\section{Administración de sesiones}
%%%%%%%%%%%%%%%%%%%%%%%%%%%%%%%%%%%%%%%%%%%%%%%%%%%%%%%%%%%%%%%%%%%%%%%%%%%%%%%

%%%%%%%%%%%%%%%%%%%%%%%%%%%%%%%%%%%%%%%%%%%%%%%%%%%%%%%%%%%%%%%%%%%%%%%%%%%%%%%
\subsection{El espacio de trabajo}
%%%%%%%%%%%%%%%%%%%%%%%%%%%%%%%%%%%%%%%%%%%%%%%%%%%%%%%%%%%%%%%%%%%%%%%%%%%%%%%

Todas las variables creadas en R se almacenan en un espacio de trabajo común.
Para ver qué variables están definidas en el área de trabajo, puede utilizar la
función \texttt{ls} (lista).  Si ha ejecutado todos los ejemplos del capítulo
anterior es posible que la salida sea similar a esto:

\begin{lstlisting}[language=R]
> ls()
 [1] "bmi"           "d"             "exp.lean"      "exp.obese"
 [5] "fpain"         "height"        "hh"            "intake.post"
 [9] "intake.pre"    "intake.sorted" "l"             "m"
[13] "mylist"        "o"             "oops"          "pain"
[17] "sel"           "v"             "weight"        "x"
[21] "xbar"          "y"
\end{lstlisting}

Recuerde no omitir los paréntesis al invocar a \texttt{ls}().

Si en algún momento las cosas comienzan a ser confusas, podemos eliminar
algunos objetos del entorno. Esto se puede hacer mediante \texttt{rm}()
(remover), de manera que:

\begin{lstlisting}[language=R]
> rm(height, weight)
\end{lstlisting}

Elimina las variables \texttt{height} y \texttt{weight}.

Se puede eliminar completamente el espacio de trabajo mediante
\texttt{rm(list=ls())}, y también mediante la opciones de menú “Remove all
objects” o “Clear Workspace” en las versiones graficas de \textbf{R}. Esto no
elimina las variables cuyo nombre comienza con un punto (\texttt{.}) por que
las mismas no son listadas por \texttt{ls()} - necesitaríamos usa
\texttt{ls(all=T)} para eso,  pero podría ser peligros ya que algunos objetos
usados por el sistema usan esta notación.

Si está familiarizado con el sistema operativo \textbf{Unix}, para el cual el
lenguaje \textbf{S}, que precedió a \textbf{R}, fue escrito originalmente,
entonces sabrá que los comandos para listar y eliminar archivos en Unix se
llaman precisamente \texttt{ls} y \texttt{rm}.

Es posible salvar completamente el espacio de trabajo en cualquier momento.
Simplemente deberemos escribir:

\begin{lstlisting}[language=R]
> save.image()
\end{lstlisting}

de esta forma, todo será salvado en un archivo de nombre \texttt{.RData} en el
directorio de trabajo actual. Las versiones gráficas también tienen un menú
para hacer esto. Cuando salimos de \textbf{R} se nos puede llegar a preguntar
si queremos salvar una imagen de nuestro espacio de trabajo; si le decimos que
sí estaremos haciendo exactamente lo mismo. Es posible también especificar un
nombre de archivo a salvar alternativo (usando comillas), también podremos
salvar objetos específicos con \texttt{save}. El archivo \texttt{.RData} se
carga automáticamente cuando se inicia \texttt{R} desde el directorio dónde se
encuentre. Otros archivos similares pueden cargarse al espacio de trabajo
mediante \texttt{load}.

%%%%%%%%%%%%%%%%%%%%%%%%%%%%%%%%%%%%%%%%%%%%%%%%%%%%%%%%%%%%%%%%%%%%%%%%%%%%%%%
\subsection{Salida Textual}
%%%%%%%%%%%%%%%%%%%%%%%%%%%%%%%%%%%%%%%%%%%%%%%%%%%%%%%%%%%%%%%%%%%%%%%%%%%%%%%

Es importante tener en cuenta que el espacio de trabajo está formado únicamente
por objetos \textbf{R}, no por ninguna de las salidas generadas durante la
sesión de trabajo.  Si desea guardar ésta utilice "Guardar en archivo" en el
menú Archivo de la versión Gráfica o utilice las funciones estándar de cortar y
pegar. También puede utilizar \textbf{ESS} (Emacs Speaks Statistics), que
funciona en todas las plataformas. Es un "modo" para el editor \textbf{Emacs}
en el que puede ejecutar toda la sesión en un búfer. Puede obtener la
\textbf{ESS} y las instrucciones de instalación de \textbf{CRAN} (ver Apéndice
A).

Una forma alternativa de redirigir la salida a un archivo es usar la función
\texttt{sink}. Se trata en gran medida de una reliquia de la época de las
terminales de 80 x 25, en la que no se disponía de técnicas de cortar y pegar,
pero que aún puede ser útil en ocasiones. En particular, se puede utilizar en
el procesamiento por lotes. La forma en que funciona es de la siguiente manera:

\begin{lstlisting}[language=R]
> sink("salidatxt")
> ls()
\end{lstlisting}

No aparece ninguna salida por pantalla!. Esto por que ahora la salida normal de
cualquier comando o función se redirige automáticamente al archivo
\texttt{salida.txt} en el directorio actual.  El sistema permanecerá trabajando
de esta forma hasta que ejecutemos nuevamente (esta vez sin parámetros):

\begin{lstlisting}[language=R]
> sink()
\end{lstlisting}

El directorio de trabajo actual se puede obtener mediante \texttt{getwd()} y
establecer otro mediante \texttt{setwd(mi directorio)}, dónde \texttt{mi
directorio} es una cadena de texto. El directorio de trabajo inicial depende
del sistema, por ejemplo para las versiones gráficas suele ser el \texttt{home
dir} del usuario y en la versión de línea de comandos es la carpeta desde dónde
se ha ejecutado \textbf{R}.

%%%%%%%%%%%%%%%%%%%%%%%%%%%%%%%%%%%%%%%%%%%%%%%%%%%%%%%%%%%%%%%%%%%%%%%%%%%%%%%
\subsection{Scripting}\label{Scripting}
%%%%%%%%%%%%%%%%%%%%%%%%%%%%%%%%%%%%%%%%%%%%%%%%%%%%%%%%%%%%%%%%%%%%%%%%%%%%%%%

Más allá de un cierto nivel de complejidad, seguramente no querrá trabajar con
\textbf{R} línea por línea. En estas situaciones, es mejor trabajar con scripts
\textbf{R}, colecciones de líneas de código almacenadas en un archivo o en la
memoria de la computadora de alguna forma.

Una opción es usar la función de \texttt{source}, que es algo así como lo
opuesto a \texttt{sink}. Toma la entrada (es decir, los comandos desde un
archivo) y los ejecuta. Nótese, sin embargo, que todo el archivo se comprueba
sintácticamente antes de que ejecutar cualquier instrucción. A menudo es útil
configurar \texttt{echo=T} en la llamada para que los comandos sean impresos
junto con la salida.

Otra opción es de naturaleza más interactiva. Puedes trabajar con un editor que
le permite enviar una o más líneas del script a una instancia de R en
ejecución, el cual se comportará como si las mismas líneas hubieran sido
introducida en el indicador de comandos. Las versiones de Windows, Linux y
Macintosh de R tienen ventanas de scripting simples incorporadas, y un número
importante de Editores de texto tienen la funcionalidad para enviar los
comandos a R.

\begin{tradnote}

	Actualmente son muchas las herramientas para trabajar en R, muchas más de
	las que existían al momento de la publicación original de este libro. En
	lineas generales hoy tenemos R en las tres principales plataformas:
	Windows, Linux y MAC, en cada una además podemos contar con:

	\begin{description}
	\item[$\bullet$] Interprete interactivo por línea de comando (\texttt{R})
	\item[$\bullet$] Versión grafica (\texttt{Rgui})
	\item[$\bullet$] Múltiples editores de texto con integración con \textbf{R}
	\item[$\bullet$] Entornos integrados de desarrollo (\textbf{Rstudio})
	\end{description}

\end{tradnote}

El historial de los comandos introducidos en una sesión se puede guardar y
volver a cargar utilizando los comandos \texttt{savehistory} y
\texttt{loadhistory}, que también se asignan a entradas de menú en la versión
gráfica. Las comandos guardados pueden ser útiles como punto de partida para
escribir scripts; observe también que la función \texttt{history()} mostrará
los últimos comandos introducidos en la consola (hasta un máximo de 25 líneas
por defecto)

%%%%%%%%%%%%%%%%%%%%%%%%%%%%%%%%%%%%%%%%%%%%%%%%%%%%%%%%%%%%%%%%%%%%%%%%%%%%%%%
\subsection{El sistema de ayuda de R}
%%%%%%%%%%%%%%%%%%%%%%%%%%%%%%%%%%%%%%%%%%%%%%%%%%%%%%%%%%%%%%%%%%%%%%%%%%%%%%%

\textbf{R} puede hacer mucho más de lo que un principiante podría llegar a
necesitar o incluso entender. Este libro está escrito de tal manera que la
mayor parte del código que probablemente necesitará en relación con los
procedimientos estadísticos se describe en el texto, y el compendio en el
Apéndice C está diseñado para proporcionar una visión general básica. Sin
embargo, es evidente que no es posible abarcar todo.

\textbf{R} también viene con una amplia ayuda en línea en forma de texto, así
como en forma de una serie de archivos HTML que se pueden leer utilizando
cualquier navegador Web Se puede acceder a las páginas de ayuda a través de
``help'' en la barra de menú de Windows e introduciendo \texttt{help.start()} en
cualquier plataforma. Encontrará que las páginas son de naturaleza técnica. La
precisión y la concisión prevalecen aquí sobre la legibilidad y la pedagogía
(algo que se aprende a apreciar después de la exposición a lo contrario).

Desde la línea de comandos, siempre puede introducir \texttt{help(aggregate)}
para obtener ayuda de la función \texttt{aggregate} o utilizar el prefijo
\texttt{?}, por ejemplo: \texttt{?aggregate}. Si el visor de HTML se está
ejecutando, entonces la página de ayuda se mostrará allí. De lo contrario, se
mostrará como texto en un misma ventana del terminal o en una ventana separada.
Tenga en cuenta que la versión HTML del sistema de ayuda incluye un ``Motor de
búsqueda y claves'' muy útil y que la función \texttt{apropos} le permite
obtener una lista de nombres de comandos que contienen un patrón dado. La
función \texttt{help.search} es similar, pero utiliza coincidencias difusas y
busca más profundamente en las páginas de ayuda, de modo que podrá localizar,
por ejemplo, el coeficiente de correlación de Kendall en \texttt{cor.test} si
utiliza \texttt{help.search("kendal")}.

También está disponible con las distribuciones de \textbf{R}, un conjunto de
documentos en varios formatos. Es particularmente interesante ``An Introduction
to R'', basado originalmente en una serie de notas para \textbf{S-PLUS} de Bill
Venables y David Smith y adaptado a \textbf{R} por varias personas.  Contiene
una introducción al lenguaje \textbf{R} y al entorno, mucho más centrada en el
lenguaje en sí que este libro. En la plataforma Windows, puede elegir instalar
documentos PDF como parte del procedimiento normal de instalación para poder
acceder luego por medio del menú ``Ayuda''.  Se puede acceder también a una
versión HTML (sin imágenes) a través del navegador en todas las plataformas.

%%%%%%%%%%%%%%%%%%%%%%%%%%%%%%%%%%%%%%%%%%%%%%%%%%%%%%%%%%%%%%%%%%%%%%%%%%%%%%%
\subsection{Paquetes}\label{paquetes}
%%%%%%%%%%%%%%%%%%%%%%%%%%%%%%%%%%%%%%%%%%%%%%%%%%%%%%%%%%%%%%%%%%%%%%%%%%%%%%%

Una instalación de \texttt{R} contiene una o más librerías o paquetes. Algunos
de estos paquetes forman parte de la instalación básica.  Otros pueden ser
descargados desde \textbf{CRAN} (ver apéndice A), el cual alberga a la fecha
más de 12000 paquetes para múltiples propósitos.  Incluso podríamos crear
nuestros propios paquetes. Una librería es por lo general una simple carpeta en
el disco. Una librería de sistema se crea al instalar \texttt{R}.  En algunas
instalaciones, no les es permitido a los usuarios modificar la librería del
sistema. De todas formas, es posible configurara librerías privadas; para más
detalles ver \texttt{help(".Library")}.

Un paquete puede contener, funciones escritas en \textbf{R}, librerías de carga
dinámica escritas en \textbf{C} o \textbf{Fortran} y distintos conjuntos de
datos.  Un paquete generalmente implementa funcionalidad que la mayoría de los
usuarios probablemente no necesiten tener cargada todo el tiempo. Un paquete es
cargado en \textbf{R} mediante el comando \texttt{library}, de esta forma, para
cargar el paquete \texttt{survival} tendremos que ingresar:

\begin{lstlisting}[language=R]
> library(survival)
\end{lstlisting}

Los paquetes cargados, no se consideran parte del espacio de trabajo. Si
terminamos la sesión \texttt{R} y comenzamos una nueva con es espacio de
trabajo guardado, deberemos cargar nuevamente los paquetes anteriores.  Por
esta misma razón, rara vez necesitemos descargar un paquete activo, pero si lo
llegaremos a necesitar hacer, podríamos escribir:

\begin{lstlisting}[language=R]
> detach("package:survival")
\end{lstlisting}

(ver sección \ref{attachdetach}.)

%%%%%%%%%%%%%%%%%%%%%%%%%%%%%%%%%%%%%%%%%%%%%%%%%%%%%%%%%%%%%%%%%%%%%%%%%%%%%%%
\subsection{Datos integrados}
%%%%%%%%%%%%%%%%%%%%%%%%%%%%%%%%%%%%%%%%%%%%%%%%%%%%%%%%%%%%%%%%%%%%%%%%%%%%%%%

Muchos paquetes, tanto dentro como fuera de la distribución R estándar, vienen
con algún que otro conjunto de datos incorporado. Estos conjuntos de datos
pueden ser bastante grandes, por lo que no es una buena idea mantenerlos en la
memoria en todo momento. Se requiere entonces, un mecanismo para la carga bajo
demanda. En muchos paquetes, esto funciona a través de un mecanismo llamado
\textit{carga perezosa} ("lazy loading,"), que permite al sistema "simular" que
los datos están sí  en memoria, pero en realidad no se cargan hasta que son
referenciados por primera vez.

Con este mecanismo, los datos están al alcance de la mano. Por ejemplo, si
escribe \texttt{thuesen}, se muestra el \texttt{data.frame} de dicho nombre.
Algunos paquetes todavía requieren llamadas explícitas a la función
\texttt{data}.  La mayoría de las veces, esto carga un \texttt{data.frame} con
el nombre que se especifica como parámetro; \texttt{data(thuesen)}, por
ejemplo, carga el \texttt{data.frame} "thuesen".

Lo que hace \texttt{data} es recorrer los directorios de datos asociados a cada
paquete (ver Sección \ref{paquetes}) y buscar archivos cuyo nombre base
coincida con el nombre especificado. Dependiendo de la extensión del archivo,
pueden suceder varias cosas. Los archivos con extensión \texttt{.tab} se leen
usando \texttt{read.table} (Sección \ref{readtextfile}), mientras que los
archivos con extensión \texttt{.R} se ejecutan como archivos de código (¡y
podrían, en general, hacer cualquier cosa!), esto simplemente, por dar dos
ejemplos comunes.

Si hay un subdirectorio del directorio actual llamado \texttt{data}, entonces
también se busca en él. Esta puede ser una forma muy útil de organizar sus
proyectos personales.

%%%%%%%%%%%%%%%%%%%%%%%%%%%%%%%%%%%%%%%%%%%%%%%%%%%%%%%%%%%%%%%%%%%%%%%%%%%%%%%
\subsection{\texttt{attach} y \texttt{detach}}\label{attachdetach}
%%%%%%%%%%%%%%%%%%%%%%%%%%%%%%%%%%%%%%%%%%%%%%%%%%%%%%%%%%%%%%%%%%%%%%%%%%%%%%%

La notación para acceder a variables en \texttt{data.frames} se vuelve bastante pesada
si tiene que escribir repetidamente comandos largos como:

\begin{lstlisting}[language=R]
> plot(thuesen$blood.glucose,thuesen$short.velocity)
\end{lstlisting}

Afortunadamente, puede hacer que\textbf{R} busque objetos entre las variables
de un \texttt{data.frame} determinado, por ejemplo \texttt{thuesen}. Escribiendo:

\begin{lstlisting}[language=R]
> attach(thuesen)
\end{lstlisting}

y entonces los datos de \texttt{thuesen} estarán disponibles sin la engorrosa
notación \texttt{\$}:

\begin{lstlisting}[language=R]
> blood.glucose
 [1] 15.3 10.8  8.1 19.5  7.2  5.3  9.3 11.1  7.5 12.2  6.7  5.2
[13] 19.0 15.1  6.7  8.6  4.2 10.3 12.5 16.1 13.3  4.9  8.8  9.5
\end{lstlisting}

Lo que sucedió aquí es que el \texttt{data.frame} \texttt{thuesen} se colocan en la
ruta de búsqueda del sistema. Puede ver la misma con el comando:

\begin{lstlisting}[language=R]
> search()
 [1] ".GlobalEnv"        "thuesen"           "package:ISwR"
 [4] "tools:rstudio"     "package:stats"     "package:graphics"
 [7] "package:grDevices" "package:utils"     "package:datasets"
[10] "package:methods"   "Autoloads"         "package:base"
\end{lstlisting}

Nótese que \texttt{thuesen} se coloca en segundo lugar en la ruta de búsqueda.
\texttt{.GlobalEnv} es el espacio de trabajo y \texttt{package:base} es la
biblioteca del sistema donde se definen todas las funciones estándar. Omitimos
describir \texttt{Autoloads}.  \texttt{package:stats} en adelante contiene las
rutinas estadísticas básicas como el test de Wilcoxon, y los otros paquetes
contienen de forma similar varias funciones y conjuntos de datos. (El sistema
de paquetes es modular, y puede ejecutar \textbf{R} con un conjunto mínimo de
paquetes para usos específicos. Por último, \texttt{package:ISwR} contiene los
conjuntos de datos utilizados para este libro.


Puede haber varios objetos del mismo nombre en diferentes partes de la ruta de
búsqueda. En ese caso, \textbf{R} elige la primera (es decir, busca primero en
\texttt{.GlobalEnv}, luego en \texttt{thuesen}, y así sucesivamente). Por esta
razón, hay que tener un poco de cuidado con los objetos "sueltos" que se
definen en el espacio de trabajo fuera de un \texttt{data.frame}, ya que se
utilizarán antes que cualquier vector y factor del mismo nombre en el
\texttt{data.frame} específico. Por la misma razón, no es una buena idea dar a
un \texttt{data.frame} el mismo nombre que a una de las variables dentro de él.
Tenga en cuenta también que cambiar un \texttt{data.frame} después de
adjuntarlo no afectará a las variables disponibles a partrir del
\texttt{attach}, ya que adjuntar implica una operación de copia (virtual) del
\texttt{data.frame}

No es posible adjuntar \texttt{data.frame} antes de \texttt{.GlobalEnv} o después de
\texttt{package:base}. Sin embargo, es posible adjuntar más de un \texttt{data.frame}. Las
nuevos \texttt{data.frame} se insertan en la posición 2 de forma predeterminada, y
todo excepto \texttt{.GlobalEnv} se mueve un paso a la derecha. Sin embargo, es posible
especificar que se debe buscar un marco de datos antes de \texttt{.GlobalEnv} utilizando
construcciones de la forma de:

\begin{lstlisting}[language=R]
> with(thuesen, plot(blood.glucose, short.velocity))
\end{lstlisting}

En algunos contextos, \textbf{R} usa un método ligeramente diferente para
buscar objetos. Si se busca una variable de un tipo específico (generalmente
una función), \textbf{R} omitirá las de otros tipos. Esto ahorra las peores
consecuencias de nombrar accidentalmente una variable, por ejemplo \texttt{c},
aunque exista una función del sistema con el mismo nombre.

Puede eliminar un \texttt{data.frame} de la ruta de búsqueda con
\texttt{detach}. Si no se dan argumentos, el \texttt{data.frame} en la posición
2 se elimina, que es generalmente lo que se desea. \texttt{.GlobalEnv} y
\texttt{package:base} no se pueden desadjuntar (\texttt{detach}) del entorno.

\begin{lstlisting}[language=R]
> detach()
> search()

 [1] ".GlobalEnv"        "package:ISwR"      "tools:rstudio"
 [4] "package:stats"     "package:graphics"  "package:grDevices"
 [7] "package:utils"     "package:datasets"  "package:methods"
[10] "Autoloads"         "package:base"
\end{lstlisting}

%%%%%%%%%%%%%%%%%%%%%%%%%%%%%%%%%%%%%%%%%%%%%%%%%%%%%%%%%%%%%%%%%%%%%%%%%%%%%%%
\subsection{subset, transform, y within}
%%%%%%%%%%%%%%%%%%%%%%%%%%%%%%%%%%%%%%%%%%%%%%%%%%%%%%%%%%%%%%%%%%%%%%%%%%%%%%%

Puede adjuntar un \texttt{data.frame} para evitar la engorrosa indexación de cada
variable dentro de él. Sin embargo, esto es poco práctico para seleccionar
subconjuntos de datos y para crear nuevos \texttt{data.frame} con variables
transformadas. Existen un par de funciones para facilitar estas operaciones. Se
utilizan de la siguiente manera:

\begin{lstlisting}[language=R]
> thue2 <- subset(thuesen,blood.glucose<7)
> thue2

   blood.glucose short.velocity
6            5.3           1.49
11           6.7           1.25
12           5.2           1.19
15           6.7           1.52
17           4.2           1.12
22           4.9           1.03

> thue3 <- transform(thuesen,log.gluc=log(blood.glucose))
> thue3

   blood.glucose short.velocity log.gluc
1           15.3           1.76 2.727853
2           10.8           1.34 2.379546
3            8.1           1.27 2.091864
4           19.5           1.47 2.970414
5            7.2           1.27 1.974081
...
22           4.9           1.03 1.589235
23           8.8           1.12 2.174752
24           9.5           1.70 2.251292
\end{lstlisting}

Note que las variables usadas en las expresiones para crear nuevas variables o
para construir subconjunto son evaluadas con variables del propio
\texttt{data.frame}.

\texttt{subset} también funciona con vectores individuales. Esto es casi lo
mismo que la indexación con un vector lógico (como ser
\texttt{short.velocity[blood.glucose<7]}), excepto que se excluyen las
observaciones con valores faltanantes (\texttt{NA}) en el criterio de
selección.

\texttt{subset} también tiene un parámetros \texttt{select} que puede usarse
para extraer variables del \texttt{data.frame}. Volveremos sobre esto en la
sección \label{appending frames}.

La función de \texttt{transform} tiene un par de inconvenientes, el más grave,
probablemente sea, que no permite cálculos encadenados donde algunas de las
nuevas variables dependen de las otras. Los signos \texttt{=} en la sintaxis no
son asignaciones, sino que indican nombres, que se asignan a los vectores
calculados en el último paso.

Una alternativa a \texttt{transform} es la función \texttt{within}, que puede
ser utilizada de esta manera:

\begin{lstlisting}[language=R]
> thue4 <- within(thuesen,{
        log.gluc <- log(blood.glucose)
        m <- mean(log.gluc)
        centered.log.gluc <- log.gluc - m
        rm(m)
})

> thue4
   blood.glucose short.velocity centered.log.gluc log.gluc
1           15.3           1.76       0.481879807 2.727853
2           10.8           1.34       0.133573113 2.379546
3            8.1           1.27      -0.154108960 2.091864
4           19.5           1.47       0.724441444 2.970414
5            7.2           1.27      -0.271891996 1.974081
...
22           4.9           1.03      -0.656737817 1.589235
23           8.8           1.12      -0.071221300 2.174752
24           9.5           1.70       0.005318777 2.251292
\end{lstlisting}

Note que el segundo parámetro es una expresión arbitraria (en este caso, una
expresión compuesta). La función es similar a \texttt{with}, pero en vezr de,
simplemente devolver el valor calculado, recoge todas las variables nuevas y
modificadas en \texttt{data.frame} modificado, el cual es devuelto.  Como se
muestra, las variables que contienen resultados intermedios pueden ser
descartadas con \texttt{rm} (es particularmente importante hacer esto si el
contenido es incompatible con el \texttt{data.frame}).

%%%%%%%%%%%%%%%%%%%%%%%%%%%%%%%%%%%%%%%%%%%%%%%%%%%%%%%%%%%%%%%%%%%%%%%%%%%%%%%
\section{El subsistema gráfico}
%%%%%%%%%%%%%%%%%%%%%%%%%%%%%%%%%%%%%%%%%%%%%%%%%%%%%%%%%%%%%%%%%%%%%%%%%%%%%%%

En la sección \ref{graficos}, vimos como crear un simple gráfico y sobre imponer
en él una curva.  Es muy común en los gráficos estadísticos que se desee crear un
gráfico que sea ligeramente diferente al predeterminado: A veces querrá añadir
anotaciones, a veces querrá que los ejes sean diferentes - etiquetas en lugar de
números, colocación irregular de marcas de verificación, etc. Todas estas cosas
se pueden realizar en \textbf{R}. Los métodos para hacerlo pueden parecer un poco
inusuales al principio, pero ofrecen un enfoque muy flexible y poderoso.

En esta sección, profundizaremos en la estructura de un gráfico o plot típico
y daremos algunas indicaciones sobre cómo puede trabajar con estos para lograr
los resultados deseados.  Tenga en cuenta, sin embargo, que se trata de un área
grande y compleja y que no está dentro del alcance de este libro cubrirla
completamente. De hecho, ignoramos completamente las nuevas herramientas
importantes en los paquetes de \texttt{grid} y \texttt{lattice}.

%%%%%%%%%%%%%%%%%%%%%%%%%%%%%%%%%%%%%%%%%%%%%%%%%%%%%%%%%%%%%%%%%%%%%%%%%%%%%%%
\subsection{Distribución de los gráficos}
%%%%%%%%%%%%%%%%%%%%%%%%%%%%%%%%%%%%%%%%%%%%%%%%%%%%%%%%%%%%%%%%%%%%%%%%%%%%%%%

En el modelo gráfico que utiliza \textbf{R}, hay (para un determinado gráfico)
una región central de graficación rodeada de márgenes. Las coordenadas dentro de
esta región de trazado se especifican en unidades de datos (el tipo utilizado
generalmente para etiquetar los ejes). Las coordenadas de los márgenes se
especifican en líneas de texto a medida que se desplaza en una dirección
perpendicular a un lado de la región de trazado, pero en unidades de datos a
medida que se desplaza por el lado. Esto es útil ya que generalmente desea poner
texto en los márgenes de un gráfico.

Un gráfico x-y estándar tiene una etiqueta de título para x e y generada a
partir de las expresiones que se están trazando. Sin embargo, puede anular estas
etiquetas y también añadir otros dos títulos, un título principal encima del
gráfico y un subtitulo en la parte inferior, en la llamada al gráfico.

\begin{lstlisting}[language=R]
> x <- runif(50,0,2)
> y <- runif(50,0,2)
> plot(x, y, main="Main title", sub="subtitle", xlab="x-label", ylab="y-label")
\end{lstlisting}

Dentro de la región de graficación, puede colocar puntos y líneas que se especifican
en la llamada a \texttt{plot} o se añaden posteriormente con \texttt{points} y
\texttt{lines}. También puede colocar un texto con:

\begin{lstlisting}[language=R]
> text(0.6,0.6,"text at (0.6,0.6)")
> abline(h=.6,v=.6)
\end{lstlisting}

En este caso, invocamos a \texttt{abline} simplemente para mostrar como el
texto está centrado en el punto ( 0.6, 0.6 ) . (Habitualmente, \texttt{abline}
dibuja la línea $y = a + bx$ cuando se la invoca con los parámetros \texttt{a}
y \texttt{b}, pero también puede usarse para dibujar una línea horizontal y
vertical tal como la hemos usado.)

Las coordenadas del margen son usadas por la función \texttt{mtext}. Tal como
podemos ver en el siguiente ejemplo:

\begin{lstlisting}[language=R]
> for (side in 1:4) mtext(-1:4,side=side,at=.7,line=-1:4)
> mtext(paste("side",1:4), side=1:4, line=-1,font=2)
\end{lstlisting}

El ciclo \texttt{for} (ver Sección \ref{flowcontrol}) coloca los números -1 al 4
alineados en cada uno de los cuatro margenes en la posición 0.7 medida en las
coordenadas del usuario.La función siguiente coloca una etiqueta en cada lado,
indicando el número del lado. El parámetro \texttt{font=2} significa que se
utiliza una fuente en negrita. Observe en la Figura \ref{fig-5} que no todos los
márgenes son lo suficientemente anchos para contener todos los números y que es
posible usar números de línea negativos para colocar el texto dentro de la
región de trazado.

%%%%%%%%%%%%%%%%%%%%%%%%%%%%%%%%%%%%%%%%%%%%%%%%%%%%%%%%%%%%%%%%%%%%%%%%%%%%%%%
\subsection{Construyendo gráficos por partes}
%%%%%%%%%%%%%%%%%%%%%%%%%%%%%%%%%%%%%%%%%%%%%%%%%%%%%%%%%%%%%%%%%%%%%%%%%%%%%%%

Los gráficos de alto nivel se componen de elementos, cada uno de los cuales
también se puede dibujar por separado. Los comandos de dibujo separados a menudo
permiten un control más fino del elemento, por lo que una estrategia estándar
para lograr un efecto dado es primero dibujar la gráfica sin ese elemento y
añadirlo posteriormente. En un caso extremo, el siguiente comando no trazará
absolutamente nada:

\begin{lstlisting}[language=R]
> plot(x, y, type="n", xlab="", ylab="", axes=F)
\end{lstlisting}

Aquí \texttt{type="n"} hace que los puntos no se dibujen. \texttt{axes=F}
suprime los ejes y el rectángulo alrededor del gráfico, y las etiquetas de
título \textit{x} e \textit{y} se establecen en cadenas vacías.

\begin{figure}[H]
  \includegraphics[width=\linewidth]{img/fig-5.pdf}
  \caption{La disposición de una gráfica estándar.}
  \label{fig-5}
\end{figure}

% pdf('doc/img/fig-5.pdf', height = 5, width = 7.5)
% x <- runif(50,0,2)
% y <- runif(50,0,2)
% plot(x, y, main="Main title", sub="subtitle", xlab="x-label", ylab="y-label")
% text(0.6,0.6,"text at (0.6,0.6)")
% abline(h=.6,v=.6)
% for (side in 1:4) mtext(-1:4,side=side,at=.7,line=-1:4)
% mtext(paste("side",1:4), side=1:4, line=-1,font=2)
% dev.off()

Sin embargo, el hecho de que no se trace ningún gráfico no significa que no haya
pasado nada. El comando configura la región de trazado y los sistemas de
coordenadas como si realmente lo hubiera hecho. Para agregar los elementos de la
gráfica, evalúe lo siguiente:

\begin{lstlisting}[language=R]
> points(x,y)
> axis(1)
> axis(2,at=seq(0.2,1.8,0.2))
> box()
> title(main="Main title", sub="subtitle",xlab="x-label", ylab="y-label")
\end{lstlisting}

Observe cómo la llamada a \texttt{axis} especifica un conjunto alternativo de
marcas de verificación (y etiquetas). Esta es una técnica común utilizada para
crear ejes especiales en un gráfico y también se puede utilizar para crear ejes
no equidistantes, así como ejes con etiquetado no numérico.

Trazar con \texttt{type="n"} es a veces una técnica útil porque tiene el efecto
secundario de dimensionar el área de la gráfica. Por ejemplo, para crear un
gráfico con diferentes colores para diferentes grupos, primero puede trazar
todos los datos con \texttt{type="n"}, asegurándose de que la región del gráfico
es lo suficientemente grande, y luego sumar los puntos para cada grupo usando
\texttt{points}. (Pasar un argumento vectorial para \texttt{col} es más
conveniente en este caso en particular.

%%%%%%%%%%%%%%%%%%%%%%%%%%%%%%%%%%%%%%%%%%%%%%%%%%%%%%%%%%%%%%%%%%%%%%%%%%%%%%%
\subsection{Utilizando \texttt{par}}
%%%%%%%%%%%%%%%%%%%%%%%%%%%%%%%%%%%%%%%%%%%%%%%%%%%%%%%%%%%%%%%%%%%%%%%%%%%%%%%

La función \texttt{par} permite un control increíblemente fino sobre los
detalles de un gráfico, aunque puede ser bastante confusa para los principiantes
(e incluso para los usuarios expertos a veces). La mejor estrategia puede ser
simplemente tratar de aprender algunos trucos útiles a la vez y de vez en cuando
tratar de resolver un problema en particular examinando la página de ayuda.

Algunos de los parámetros, pero no todos, también se pueden establecer mediante
parámetros a las funciones de dibujo, que también tienen algunos parámetros que
no se pueden establecer mediante \texttt{par}.  Cuando un parámetro se puede
establecer por ambos métodos, la diferencia es, generalmente que si algo se se
configuró mediante \texttt{par}, entonces permanecerá configurado
posteriormente.

La configuración de \texttt{par} permite controlar el ancho y el tipo de línea,
así como el tamaño de los caracteres y fuente de la letra, color, estilo de
cálculo de los ejes, tamaño de la gráfica y figuras, recortes, etc. Es posible
dividir una figura en varias subfiguras utilizando los parámetros \texttt{mfrow}
y \texttt{mfcol}.

Por ejemplo, los tamaños de margen predeterminados son algo más de 5, 4, 4 y 2
líneas, sin embargo es posible modificar esto definiendo \texttt{par(mar=c(4,4,2,2)+0.1)} antes de graficar. Esto
quita una línea fuera del margen inferior y dos líneas fuera del margen superior
del gráfico, lo que reducirá la cantidad de espacios en blanco no utilizados
cuando no haya un título o subtítulo principal. Si se mira cuidadosamente, se
notará que la Figura \ref{fig-5} tiene una región de trazado algo más
pequeña que las otras gráficas de este libro. Esto se debe a que las otras
gráficas se han realizado con márgenes reducidos por razones de composición
tipográfica.

Sin embargo, no tiene sentido describir los parámetros gráficos de forma
completa en este momento. Ya volveremos a ellos, cuando los utilicemos
para gráficos específicos.

%%%%%%%%%%%%%%%%%%%%%%%%%%%%%%%%%%%%%%%%%%%%%%%%%%%%%%%%%%%%%%%%%%%%%%%%%%%%%%%
\subsection{Combinando gráficos}\label{combigraf}
%%%%%%%%%%%%%%%%%%%%%%%%%%%%%%%%%%%%%%%%%%%%%%%%%%%%%%%%%%%%%%%%%%%%%%%%%%%%%%%

Algunas consideraciones especiales surgen cuando usted desea poner varios
elementos juntos en la misma gráfica. Considere la posibilidad de superponer un
histograma de una densidad normal (consulte las Secciones 4.2 y 4.4.1 para
obtener información sobre los histogramas y la Sección 3.5.1 para la densidad).
Lo siguiente está cerca, pero no es lo suficientemente bueno (figura no
mostrada).

\begin{lstlisting}[language=R]
> x <- rnorm(100)
> hist(x,freq=F)
> curve(dnorm(x),add=T)
\end{lstlisting}

El parámetro \texttt{feq=F} en la llamada a \texttt{hist} nos asegura que el
histograma estará en términos de densidades en lugar de conteos absolutos.  La
función \texttt{curve} grafíca una expresión (en términos de x) y \texttt{add=T}
permite sobreponer la misma a una gráfica ya existente.  Por lo tanto, las cosas
parecieran estar configuradas correctamente, pero a veces la parte superior de
la función de densidad se corta. La razón es, por supuesto, que la altura de la
densidad normal no jugó ningún papel en el ajuste del eje \texttt{y} para el
histograma. No ayuda nada,  invertir el orden y dibujar la curva primero y
agregar el histograma luego, porque entonces las barras más altas podrían ser
las recortadas.

La solución es primero obtener la magnitud de los valores de \texttt{y} para ambos
elementos de la gráfica y hacer luego  que la gráfica sea lo suficientemente grande
como para contener ambos (Figura 2.2):

\begin{lstlisting}[language=R]
h <- hist(x, plot=F)
ylim <- range(0, h$density, dnorm(0))
hist(x, freq=F, ylim=ylim)
curve(dnorm(x), add=T)
\end{lstlisting}

\begin{figure}[H]
  \includegraphics[width=\linewidth]{img/fig-6.pdf}
  \caption{Histograma con densidad normal superpuesta.}
  \label{fig:fig-6}
\end{figure}

% pdf('doc/img/fig-6.pdf', height = 5, width = 7.5)
% x <- rnorm(100)
% hist(x,freq=F)
% curve(dnorm(x),add=T)
% dev.off()

Cuando invocamos \texttt{hist} con \texttt{plot=F}, no se trazará ningún
gráfico, pero retornará una estructura que contiene las alturas de las barras en
la escala de densidad. Esto y el hecho de que el máximo de \texttt{dnorm(x)} sea
\texttt{dnorm(0)} nos permite calificar un rango que cubre tanto las barras como
la densidad normal. El cero en el rango de llamada asegura que la parte inferior
de las barras también estará dentro del rango.  El rango de valores de
\texttt{y} se pasa a la función \texttt{hist} mediante el parámetro
\texttt{ylim}.

%%%%%%%%%%%%%%%%%%%%%%%%%%%%%%%%%%%%%%%%%%%%%%%%%%%%%%%%%%%%%%%%%%%%%%%%%%%%%%%
\section{Programando en \textbf{R}}
%%%%%%%%%%%%%%%%%%%%%%%%%%%%%%%%%%%%%%%%%%%%%%%%%%%%%%%%%%%%%%%%%%%%%%%%%%%%%%%

Es posible escribir nuestras propias funciones en \textbf{R}. De hecho, esto es
un gran atractivo para trabajar con el sistema a largo plazo. Este libro evita
en gran medida este tema a favor de cubrir un conjunto más amplio de
procedimientos estadísticos básicos que pueden ejecutarse desde la línea de
comandos. Sin embargo, para darle una idea de lo que se puede hacer, considere
la siguiente función, que contiene el código del ejemplo de la Sección
\ref{combigraf} escrita para que pueda decir invocarla sencillamente así:
 \texttt{hist.with.normal(rnorm(200))}. Se ha ampliado ligeramente, de modo que
ahora utiliza la media empírica y la desviación estándar de los datos en lugar
de sólo 0 y 1.

\begin{lstlisting}[language=R]
> hist.with.normal <- function(x, xlab=deparse(substitute(x)),...) {
+     h <- hist(x, plot=F, ...)
+     s <- sd(x)
+     m <- mean(x)
+     ylim <- range(0,h$density,dnorm(0,sd=s))
+     hist(x, freq=F, ylim=ylim, xlab=xlab, ...)
+     curve(dnorm(x,m,s), add=T)
+ }
\end{lstlisting}

Note el uso de un parámetro por defecto para \texttt{xlab}. Si no se especifica
\texttt{xlab}, la etiqueta del eje \texttt{x}, entonces se obtiene la expresión
indicada en el parámetro \texttt{x} para transformarla luego en una cadena y así
poder usarla; esto quiere decir, que si le pasamos \texttt{rnorm(100)} al
parámetro \texttt{x} entonces la etiqueta para el eje se convertirá en
\texttt{''rnom(100)''}. Observe también el uso de un parámetro \texttt{...}, que
recoge cualquier parámetro  y lo pasa a las dos llamadas a la función
\texttt{hist}.

Se puede aprender más sobre programación en \textbf{R} estudiando las funciones
incorporadas, empezando por las sencillas como \texttt{log10} o
\texttt{weighted.mean}.

%%%%%%%%%%%%%%%%%%%%%%%%%%%%%%%%%%%%%%%%%%%%%%%%%%%%%%%%%%%%%%%%%%%%%%%%%%%%%%%
\subsection{Control de flujo} \label{flowcontrol}
%%%%%%%%%%%%%%%%%%%%%%%%%%%%%%%%%%%%%%%%%%%%%%%%%%%%%%%%%%%%%%%%%%%%%%%%%%%%%%%

Hasta ahora, hemos visto componentes del lenguaje \textbf{R} que causan la
evaluación de expresiones individuales. Sin embargo, \textbf{R} es un verdadero
lenguaje de programación que permite la ejecución condicional y las
construcciones cíclicas también. Considere, por ejemplo, el siguiente código.
(El código implementa una versión del método de Newton para calcular la raíz
cuadrada de y.)

\begin{lstlisting}[language=R]
y <- 12345
x <- y/2
while (abs(x*x-y) > 1e-10) x <- (x + y/x)/2
x
[1] 111.1081
x^2
\end{lstlisting}

Fíjense en la construcción \texttt{while(condición) expresión}, que dice
que la expresión debe ser evaluada siempre y cuando la condición sea
\texttt{VERDADERA}. La prueba se realiza en la parte superior del bucle, por lo
que la expresión podría no ser evaluada nunca.

Una variación del mismo algoritmo con la condición en la parte inferior del
bucle se puede escribir con una construcción de repetición:

\begin{lstlisting}[language=R]
> x <- y/2
> repeat{
+     x <- (x + y/x)/2
+     if (abs(x*x-y) < 1e-10) break
+ }
> x
[1] 111.1081
\end{lstlisting}

Esto también ilustra otras tres estructuras de control del flujo: a) una
\textit{expresión compuesta}, que son varias expresiones unidas entre llaves; b)
una construcción \texttt{if} para una ejecución condicional; y c) una exprexión
\texttt{break}, que hace que el bucle finalice.

Por cierto, el bucle podría permitir que \texttt{y} sea un vector simplemente
cambiando la condición de terminación a:

\begin{lstlisting}[language=R]
if (all(abs(x*x - y) < 1e-10)) break
\end{lstlisting}

Esto iteraría excesivamente por algunos elementos, pero la aritmética
vectorizada probablemente compensaría con creces eso. Sin embargo, la
construcción de bucle más frecuentemente utilizada es pael \texttt{for}, que
construye un bucle sobre un conjunto fijo de valores como en el siguiente
ejemplo, que traza un conjunto de curvas de potencia en el intervalo unitario.

\begin{lstlisting}[language=R]
> x <- seq(0, 1,.05)
> plot(x, x, ylab="y", type="l")
> for ( j in 2:8 ) lines(x, x^j)
\end{lstlisting}

Observa la \textit{variable de ciclo} \texttt{j}, que toma un valor de la
secuencia con cada vuelta del ciclo, para usarla en la llamada a \texttt{lines}.

%%%%%%%%%%%%%%%%%%%%%%%%%%%%%%%%%%%%%%%%%%%%%%%%%%%%%%%%%%%%%%%%%%%%%%%%%%%%%%%
\subsection{Clases y funciones genéricas}
%%%%%%%%%%%%%%%%%%%%%%%%%%%%%%%%%%%%%%%%%%%%%%%%%%%%%%%%%%%%%%%%%%%%%%%%%%%%%%%

La programación orientada a objetos consiste en crear sistemas coherentes de
datos y métodos que funcionen sobre ellos. Uno de los propósitos es simplificar
los programas, acomodando el hecho de que tendrá métodos conceptualmente
similares para diferentes tipos de datos, aunque las implementaciones terminen
siendo diferentes. Un ejemplo de clásico, es el \textit{método print}: Tiene
sentido imprimir muchos tipos de objetos de datos, pero el diseño de la
impresión dependerá de lo que sea cada objeto. Generalmente tendrá una
\textit{clase} de objetos de datos y un \textit{método print} para esa clase.
Hay varios lenguajes orientados a objetos que implementan estas ideas de
diferentes maneras.

La mayor parte del código básico de \textbf{R} utiliza el mismo sistema de
objetos que la versión 3 de \textbf{S}. En los últimos años se ha desarrollado
un sistema de objetos alternativo similar al de la versión 4 de \textbf{S}. El
nuevo sistema tiene varias ventajas sobre el antiguo, pero limitaremos la
atención a este último. El sistema de objetos \textbf{S3} es un sistema simple
en el que un objeto tiene un atributo de \texttt{class}, que es simplemente un
vector de carácter. Un ejemplo de ello es que todos los valores de retorno de
las pruebas clásicas como \texttt{t.test} tienen la clase \texttt{"htest"}, lo
que indica que son el resultado de una prueba de hipótesis. Cuando se imprimen
estos objetos, se hace mediante \texttt{print.htest}, que crea una bonita salida
(véase el capítulo 5 para más ejemplos). Sin embargo, desde un punto de vista
programático, estos objetos son sólo listas, y puedes, por ejemplo, extraer el
\textit{p}-valor escribiendo

\begin{lstlisting}[language=R]
> t.test(bmi, mu=22.5)$p.value
[1] 0.7442183
\end{lstlisting}

La función \texttt{print} es una función generica (\textit{generic function}),
un tipo de función que actua de forma diferente de acuerdo al parámetro inical.
Esto generalmente se ve así:

\begin{lstlisting}[language=R]
> print
function (x, ...)
UseMethod("print")
<environment: namespace:base>
\end{lstlisting}

\texttt{UseMethod("print")} significa que \textbf{R} le va a pasar el control a
una función cuyo nombre depende de la clase de objeto (\texttt{print.htest} para
los objetos de la clase \texttt{"htest"}, etc.) o, en caso de no encontrarse, se
invocará \texttt{print.default}. Para ver todos los métodos disponibles para
\texttt{print}, escriba \texttt{methods(print)} (hay unos 138 disponibles en
\textbf{R} 2.6.2, demasiados para mostrarlos aquí)

%%%%%%%%%%%%%%%%%%%%%%%%%%%%%%%%%%%%%%%%%%%%%%%%%%%%%%%%%%%%%%%%%%%%%%%%%%%%%%%
\section{Entrada de datos}
%%%%%%%%%%%%%%%%%%%%%%%%%%%%%%%%%%%%%%%%%%%%%%%%%%%%%%%%%%%%%%%%%%%%%%%%%%%%%%%

Para ingresar datos mediante \texttt{c(...)}, estos no deben ser demasiado
grandes ya que resultaría impracticable cargarlos de esta forma. La mayoría de
los ejemplos de este libro utilizan conjuntos de datos incluidos en el paquete
\texttt{ISwR}, puesto a su disposición por la biblioteca (\texttt{ISwR}). Sin
embargo, tan pronto como usted desee aplicar los métodos a sus propios datos,
tendrá que ocuparse de los formatos de los archivos de datos y de su
especificación. En esta sección discutiremos cómo leer los archivos de datos y
cómo usar el módulo de edición de datos en \textbf{R}. El texto tiene cierto
sesgo hacia los sistemas Windows, principalmente debido a algunas cuestiones
especiales que deben ser mencionadas para esa plataforma


%%%%%%%%%%%%%%%%%%%%%%%%%%%%%%%%%%%%%%%%%%%%%%%%%%%%%%%%%%%%%%%%%%%%%%%%%%%%%%%
\subsection{Leyendo desde un archivo de texto} \label{readtextfile}
%%%%%%%%%%%%%%%%%%%%%%%%%%%%%%%%%%%%%%%%%%%%%%%%%%%%%%%%%%%%%%%%%%%%%%%%%%%%%%%
La forma más conveniente de leer datos en R es a través de la función
\texttt{read.table}. Requiere que los datos estén en "formato ASCII"; es decir,
un "archivo plano" como el que se crea con el NotePad de Windows o cualquier
otro editor de texto plano. El resultado de \texttt{read.table} es un "data
frame", y se espera encontrar los datos de forma tal que cada línea del archivo
contenga todos los datos de un sujeto (u observación) en un orden específico,
separados por espacios en blanco u, opcionalmente, algún otro separador. La
primera línea del archivo puede contener un encabezado con los nombres de las
variables, una práctica muy recomendada.

La tabla 11.6 de Altman (1991) contiene un ejemplo sobre la velocidad de
acortamiento de la circunferencia ventricular frente a la glucosa en sangre en
ayunas por Thuesen et Al. Utilizamos esos datos para ilustrar el subconjunto y
los utilizamos de nuevo en el capítulo sobre la correlación y la regresión.
Están entre los datos incorporados conjuntos en el paquete \texttt{ISwR} y
disponibles como el marco de datos \texttt{thuesen}, pero el punto aquí es
mostrar cómo leerlos desde un archivo de texto simple.

\begin{lstlisting}[language=R]
blood.glucose short.velocity
15.3          1.76
10.8          1.34
8.1           1.27
19.5          1.47
7.2           1.27
5.3           1.49
9.3           1.31
11.1          1.09
7.5           1.18
12.2          1.22
6.7           1.25
5.2           1.19
19.0          1.95
15.1          1.28
6.7           1.52
8.6           NA
4.2           1.12
10.3          1.37
12.5          1.19
16.1          1.05
13.3          1.32
4.9           1.03
8.8           1.12
9.5           1.70
\end{lstlisting}

Para introducir los datos en el archivo, se puede iniciar el NotePad de Windows
o cualquier otro editor de texto plano, como los que se comentan en la
Sección \ref{Scripting}. Los usuarios de Unix/Linux deberían usar sólo un editor
estándar, como \texttt{emacs} o \texttt{vi}. Si es necesario, puede incluso
utilizar un programa de procesamiento de textos con un poco de cuidado.

Simplemente debe escribir los datos como se muestra. Observe que las columnas
están separadas por un número arbitrario de espacios en blanco y que \texttt{NA}
representa un valor faltante.

Al final, debería guardar los datos en un archivo de texto. Tenga en cuenta que
los procesadores de texto requieren acciones especiales para poder guardarlos
como texto. Su formato de guardado normal podría llegar a ser difícil de leer en
otros programas.

Asumiendo además que el archivo está en la carpeta \texttt{ISwR} de la unidad
\texttt{N:}, el los datos pueden ser leídos mediante:

\begin{lstlisting}[language=R]
> thuesen2 <- read.table("N:/ISwR/thuesen.txt",header=T)
\end{lstlisting}

Observe que \texttt{header=T} especifica que la primera línea es un encabezado que
contiene los nombres de las variables dentro del archivo. También note que
se usan barras inclinadas (/), no barras invertidas (\textbackslash), en el nombre del
archivo, incluso en un sistema Windows.

La razón para evitar las barras invertidas en los nombres de archivos de
Windows, es que el símbolo se utiliza como un carácter de escape (véase la
Sección \ref{encomillado}) y, por lo tanto, necesita duplicarlo. Podría haber
usado:

\begin{lstlisting}[language=R]
> thuesen2 <- read.table("N:\\ISwR\\thuesen.txt",header=T)
\end{lstlisting}

El resultado es un \textit{data frame}, que se asigna a la variable
\texttt{thuesen2} y tiene el siguiente aspecto:

\begin{lstlisting}[language=R]
> thuesen2
   blood.glucose short.velocity
1           15.3           1.76
2           10.8           1.34
3            8.1           1.27
4           19.5           1.47
5            7.2           1.27
6            5.3           1.49
7            9.3           1.31
8           11.1           1.09
9            7.5           1.18
10          12.2           1.22
11           6.7           1.25
12           5.2           1.19
13          19.0           1.95
14          15.1           1.28
15           6.7           1.52
16           8.6             NA
17           4.2           1.12
18          10.3           1.37
19          12.5           1.19
20          16.1           1.05
21          13.3           1.32
22           4.9           1.03
23           8.8           1.12
24           9.5           1.70
\end{lstlisting}

Para leer las variables como factores (véase la sección \ref{factors}), la forma más
fácil puede ser codificarlas mediante una representación textual. La función
\texttt{read.table} detecta automáticamente si un vector es de texto o numérico y lo
convierte en un factor en el primer caso (pero no intenta reconocer los factores
codificados en forma numérica).Por ejemplo, el conjunto de datos incorporados de
\texttt{secretin} se lee de un archivo que comienza así:

\begin{lstlisting}[language=R]
   gluc person time repl time20plus time.comb
1    92      A  pre    a        pre       pre
2    93      A  pre    b        pre       pre
3    84      A   20    a        20+        20
4    88      A   20    b        20+        20
5    88      A   30    a        20+       30+
6    90      A   30    b        20+       30+
7    86      A   60    a        20+       30+
8    89      A   60    b        20+       30+
9    87      A   90    a        20+       30+
10   90      A   90    b        20+       30+
11   85      B  pre    a        pre       pre
12   85      B  pre    b        pre       pre
13   74      B   20    a        20+        20
...
\end{lstlisting}

Este archivo puede ser leído directamente por \texttt{read.table} sin más argumentos que
el nombre del archivo. Reconocerá el caso en que la primera línea sea un
elemento más corto que el resto e interpretará que la disposición implica que la
primera línea contiene un encabezamiento y el primer valor de todas las líneas
subsiguientes es una etiqueta de fila, es decir, exactamente la disposición
generada al imprimir un \textit{data frame}.

La lectura de factores como este puede ser conveniente, pero hay un
problema: el orden de los niveles es alfabético, así que por ejemplo:

\begin{lstlisting}[language=R]
> levels(secretin$time)
[1] "20" "30" "60" "90" "pre"
\end{lstlisting}

Si esto no es lo que quiere, entonces puede que tenga que manipular los niveles
de factor; véase la sección \ref{maipulate levels}.

\begin{tradnote}En la versión 4.0.0 se implementó un cambio muy significativo
con respecto de los factores: a partir de esta versión las funciones que
interpretan las cadenas com factores, tal como lo hace \texttt{readtable} dejan
de hacerlo, y se implementa por defecto la opción \texttt{stringsAsFactors =
FALSE}.
\end{tradnote}

Una nota técnica: Los archivos mencionados anteriormente están contenidos en el
paquete \texttt{ISwR} en el subdirectorio \texttt{rawdata}. La ubicación exacta
del archivo en su sistema dependerá de dónde se instaló el paquete
\texttt{ISwR}. Puedes averiguarlo de la siguiente manera:

\begin{lstlisting}[language=R]
> system.file("rawdata", "thuesen.txt", package="ISwR")
[1] "/home/pd/Rlibrary/ISwR/rawdata/thuesen.txt"
\end{lstlisting}

%%%%%%%%%%%%%%%%%%%%%%%%%%%%%%%%%%%%%%%%%%%%%%%%%%%%%%%%%%%%%%%%%%%%%%%%%%%%%%%
\subsection{Más detalles sobre \texttt{read.table}}
%%%%%%%%%%%%%%%%%%%%%%%%%%%%%%%%%%%%%%%%%%%%%%%%%%%%%%%%%%%%%%%%%%%%%%%%%%%%%%%

La función \texttt{read.table} es una herramienta muy flexible que se controla
con muchas opciones. No intentaremos hacer una descripción completa aquí, pero
sí dar una indicación de lo que se puede hacer con ella.

\textit{Detalles del formato del archivo}

Ya hemos visto el uso de \texttt{header=T}. Hay muchas otras opciones que
controlan el formato detallado del archivo de entrada:

\begin{itemize}

\item
\textbf{Separador de campo.} Esto se puede especificar usando el parámetro
\texttt{sep}. Tenga en cuenta que cuando se usa esto, como en oposición al uso
predeterminado de espacios en blanco, debe haber exactamente un separador entre
todos los campos de datos. Dos separadores consecutivos implicarán que hay un
valor perdido en el medio. Por el contrario, es necesario usar códigos
específicos para representar valores perdidos en el formato predeterminado y
también para usar alguna forma de citado de cadenas que contienen espacios
incrustados.

\item
\textbf{Cadenas NA.} Puede especificar qué cadenas del archivo representan
valores perdidos por medio de \texttt{na.strings}. Puede haber varias cadenas
diferentes, aunque no cadenas diferentes para diferentes columnas. Para imprimir
archivos del programa SAS, debería usar  \texttt{na.strings = "."}.

\item
\textbf{Citas y comentarios.} Por defecto, las comillas de estilo \textbf{R} se
pueden usar para delimitar las cadenas y partes del archivo que siguen al
carácter de comentario \texttt{\#} se ignoran. Estas características se pueden
modificar o eliminar a través de los parámetros \texttt{quote} y
\texttt{comment.char}

\item
\textbf{Número de columnas desigual.} Normalmente se considera un error si todas
las líneas no contienen la misma cantidad de columnas (la primera línea puede
tener una columna menos, como vió antes con los datos \texttt{secretin}). Los
parámetros \texttt{fill} y \texttt{flush} puede usarse en el caso que la líneas
varíen la cantidad de columnas.

\end{itemize}

\textit{Tipos de archivo delimitados}

Aplicaciones como las hojas de cálculo y bases de datos, producen archivos de
texto en formatos que requieren múltiples opciones para ajustar. Para tal fin,
existen variantes "prearmadas" de \texttt{read.table}. Dos de estas, están
destinadas a manejar archivos CSV y se laman \texttt{read.csv} y
\texttt{read.csv2}. La primer variante supone que los campos están separados por
una coma, y la última, que están separados por punto y coma pero usan una coma
como punto decimal (este formato es a menudo generado en regiones europeas o
latinas). Ambas variantes tienen \texttt{header = T} por defecto. Otras
variantes son \texttt{read.delim} y \texttt{read.delim2} para leer archivos
delimitados (por defecto delimitados por tabuladores).

\textit{Conversión de entrada}

Puede ser conveniente anular los mecanismos de conversión predeterminados de
\texttt{read.table}. Por defecto, la entrada no numérica se convierte en
factores, pero no siempre esto tiene sentido. Por ejemplo, los nombres y las
direcciones no deberían convertirse. Esto se puede modificarse para todas las
columnas de cadena de caracteres mediante el parámetro
\texttt{stringsAsFactors} o solo para algunas en particular, mediante
\texttt{as.is}.

la conversión automática es a menudo conveniente, pero es ineficiente en cuanto
a tiempos de proces y almacenamiento; para leer una columna numérica
\texttt{read.table} primero la lee como si fuera una cadena de caracteres,
comprueba si todos los elementos se pueden convertir a un  número y solo
entonces, realiza la conversión. El parámetro \texttt{colClasses} le permite
omitir este mecanismo, especificando de forma explícita, qué columnas son, de
qué clase (Las clases estándar \texttt{''character''}, \texttt{''numeric''},
etc., reciben un trato especial). También puede ignorar columnas al especificar
\texttt{''NULL''} como la clase

%%%%%%%%%%%%%%%%%%%%%%%%%%%%%%%%%%%%%%%%%%%%%%%%%%%%%%%%%%%%%%%%%%%%%%%%%%%%%%%
\subsection{El editor de datos}
%%%%%%%%%%%%%%%%%%%%%%%%%%%%%%%%%%%%%%%%%%%%%%%%%%%%%%%%%%%%%%%%%%%%%%%%%%%%%%%

\texttt{R} le permite editar los \textit{data frames} utilizando una interfaz
similar a una hoja de cálculo. La interfaz es un poco tosca pero bastante útil
para pequeños conjuntos de datos.

Para editar un \textit{data frame}, hay que usar la función \texttt{edit}:

\begin{lstlisting}[language=R]
> aq <- edit(airquality)
\end{lstlisting}


Esto muestra un editor similar a una hoja de cálculo con una columna para cada
variable en el \textit{data frame}. El conjunto de datos \texttt{airquality}
está integrado en R; revise \texttt{help(airquality)} por estudiar su contenido.
Dentro del editor, puede moverse con el mouse o las teclas de cursor y editar la
celda actual ingresando datos en ella. El tipo de variable puede modificarse
entre real (numérico) y carácter (factor) haciendo clic en la columna
encabezado, y el nombre de la variable se puede cambiar de manera similar. Tenga
en cuenta que (al menos hasta la versió 2.6.2)  ninguna manera para eliminar
filas y columnas y que los nuevos datos pueden ser ingresados solo al final.

Cuando cierra el editor, el \textit{data frame} editado se asigna a la variable
\texttt{aq}. Los datos originales, el \textit{data frame} \texttt{airquality},
permanecerán intactos. Si no le preocupa, puede sobre escribir el \textit{data
frame} original:

\begin{lstlisting}[language=R]
> fix(aq)
\end{lstlisting}

Lo que es equivalente a

\begin{lstlisting}[language=R]
> aq <- edit(aq)
\end{lstlisting}

Para ingresar datos en un \textit{data frame} en blanco:

\begin{lstlisting}[language=R]
> dd <- data.frame()
> fix(dd)
\end{lstlisting}

Una alternativa sería \texttt{dd <- edit(data.frame())}, que funciona bien
excepto que los principiantes tienden a volver a ejecutar el comando cuando
necesitan editar \texttt{dd} de nuevo, lo que por supuesto, destruye todos los
datos. En cualquier caso, es necesario comenzar con un \textit{data frame} vacío
ya que cuando a \texttt{edit} se lo invoca in prámetros, se sume que lo que se
busca es editar un función nueva, por lo se inicia un ventana de edición de
texto.

%%%%%%%%%%%%%%%%%%%%%%%%%%%%%%%%%%%%%%%%%%%%%%%%%%%%%%%%%%%%%%%%%%%%%%%%%%%%%%%
\subsection{Interfaz con otros programas}
%%%%%%%%%%%%%%%%%%%%%%%%%%%%%%%%%%%%%%%%%%%%%%%%%%%%%%%%%%%%%%%%%%%%%%%%%%%%%%%

En algún momento neceitará mover datos entre \textbf{R} y otros paquetes
estadísticos u hojas de cálculo Un enfoque sencillo y seguro, es solicitar que
el paquete en cuestión exporte los datos como un archivo de texto de algún tipo
y usar \texttt{read.table}, \texttt{read.csv}, \texttt{read.csv2},
\texttt{read.delim} o \texttt{read.delim2}, como se describió anteriormente.

El paquete \texttt{foreign}, es uno de esos, que podríamos etiquetar como
"recomendado", disponible en las distribuciones binarias de \textbf{R}. Contiene
rutinas para leer archivos en diferentes formatos, incluye aquellos como SPSS
(formato \texttt{.sav}), SAS (librerías de exportación), Epi-Info
(\texttt{.rec}), Stata, Systat, Minitb, y alguna de las versiones de S-PLUS de
los archivos de volcado.

Los usuarios de Unix/Linux a veces se encuentran con datos escritos en máquinas
con Windows. El paquete \texttt{foreign} funcionará allí también para los
formatos que soporta. Note que los datos normales de SAS no están entre los
formatos soportados. Estos tienen que ser convertidos a librerías de exportación
en el sistema origen. Los datos que han sido introducidos en hojas de cálculo de
Microsoft Excel se extraen más convenientemente usando una aplicación compatible
como OOo (OpenOffice.org).


Una técnica conveniente es leer desde el portapapeles del sistema. Digamos,
resaltar una región rectangular eden una hoja de cálculo, presionar Ctrl-C (si
está en Windows), y dentro de R usar:

\begin{lstlisting}[language=R]
read.table("clipboard", header=T)
\end{lstlisting}

Aunque esto requiere tener un poco de cuidado, ya que puede resultar en una
pérdida de precisión debido a que sólo se transfieren los datos tal como
aparecen en la pantalla. Esto es mayormente un problema si tienes datos con
muchos dígitos significativos.

Para los datos almacenados en las bases de datos, existen varios paquetes de
interfaz en CRAN. En particular para Windows y otras bases de datos en
Unix, es el paquete RODBC, porque puede configurar conexiones ODBC (``Open
Database Connectivity'') con datos almacenados por aplicaciones comunes,
incluyendo Excel y Access. Algunas bases de datos Unix (por ejemplo, PostgreSQL)
también permiten conexiones ODBC.

Para obtener información actualizada sobre estos temas, consulte el manual
``R Data Import/Export'' que viene con el sistema.


%%%%%%%%%%%%%%%%%%%%%%%%%%%%%%%%%%%%%%%%%%%%%%%%%%%%%%%%%%%%%%%%%%%%%%%%%%%%%%%
\section{Ejercicios}
%%%%%%%%%%%%%%%%%%%%%%%%%%%%%%%%%%%%%%%%%%%%%%%%%%%%%%%%%%%%%%%%%%%%%%%%%%%%%%%

2.1 Describa cómo insertar un valor entre dos elementos de un vector en una determinada
posición usando la función \texttt{append} (use el sistema de ayuda para averiguarlo).
Sin usr \texttt{append}, ¿cómo lo haría?

2.2 Escriba el conjunto de datos \texttt{thuessen} a un archivo de texto
separado por tabulaciones mediante la función \texttt{write.table}. Visualícelo con un editor de
texto (según su sistema). Modifique el valor \texttt{NA} por \texttt{.} (punto)
y lea el archivo modificado nuevamente en \textbf(R) con la función apropiada.
Intente también importar los datos a otras aplicaciones de su elección y
exportándolos a un nuevo archivo después de editarlos. Puede que tenga que
eliminar los nombres de fila para hacer que esto funcione.

%%%%%%%%%%%%%%%%%%%%%%%%%%%%%%%%%%%%%%%%%%%%%%%%%%%%%%%%%%%%%%%%%%%%%%%%%%%%%%%
\chapter{Probabilidad y distribuciones}
%%%%%%%%%%%%%%%%%%%%%%%%%%%%%%%%%%%%%%%%%%%%%%%%%%%%%%%%%%%%%%%%%%%%%%%%%%%%%%%

Los conceptos de aleatoriedad y probabilidad son fundamentales para las
estadísticas. Es un hecho empírico que la mayoría de los experimentos e
investigaciones no son perfectamente reproducibles. El grado de
irreproducibilidad puede variar: Algunos experimentos en física pueden producir
datos que son exactos hasta muchos decimales, mientras que los datos sobre
sistemas biológicos son típicamente mucho menos confiables. Sin embargo, la
visión de los datos como algo que proviene de una distribución estadística es
vital para entender los métodos estadísticos. En esta sección se esbozan las
ideas básicas de probabilidad y las funciones que tiene \textbf{R} para el
muestreo aleatorio y el manejo de las distribuciones teóricas.

%%%%%%%%%%%%%%%%%%%%%%%%%%%%%%%%%%%%%%%%%%%%%%%%%%%%%%%%%%%%%%%%%%%%%%%%%%%%%%%
\section{Muestreo aleatorio}
%%%%%%%%%%%%%%%%%%%%%%%%%%%%%%%%%%%%%%%%%%%%%%%%%%%%%%%%%%%%%%%%%%%%%%%%%%%%%%%

Gran parte de los primeros trabajos en la teoría de la probabilidad se referían
a juegos y apuestas, basados en consideraciones de simetría. Por lo tanto, la
noción básica de una  muestra aleatoria es la de repartir de una baraja bien
barajada o recoger bolas numeradas de una urna bien revuelta.

En \textbf{R}, se pueden simular estas situaciones con la función
\texttt{sample}. Si quieres elegir cinco números al azar del conjunto de
\texttt{1:40}, entonces se puede escribir:

\begin{lstlisting}[language=R]
> sample(1:40,5)
[1] 36 37 26 24  3
\end{lstlisting}

El primer parámetro (\texttt{x}) es el vector de valores posibles de dónde
obtendremos la muestra y el segundo (\texttt{size}) es el tamaño de la muestra.
En realidad, \texttt{sampe(40,5)} sería suficiente ya que un solo número se
interpreta como la longitud de una secuencia de números enteros desde el 1.

Note que el comportamiento predeterminado de \texttt{sample} es el de
\textit{muestreo sin reemplazo}. Es decir, las muestras no contendrán ningún
número repetido, y \texttt{size} obviamente no puede ser mayor que la longitud
del vector a muestrear. Si quiere muestreo con reemplazo, entonces necesita
agregar el parámetro \texttt{replace=TRUE}.

El muestreo con reemplazo es adecuado para modelar lanzamientos de monedas o
lanzamientos de un dado. Así, por ejemplo, para simular 10 lanzamientos de
monedas podríamos escribir:

\begin{lstlisting}[language=R]
> sample(c("H","T"), 10, replace=T)
[1] "T" "T" "T" "T" "T" "H" "H" "T" "H" "T"
\end{lstlisting}

En un lanzamiento normal de monedas, la probabilidad de cara debe ser igual a la
probabilidad de cruz, pero la idea de un evento aleatorio no se limita a casos
simétricos.  Podría aplicarse igualmente bien a otros casos, como el resultado
exitoso de un procedimiento quirúrgico. Esperemos que haya más de un 50\% de
posibilidades de que esto suceda. Puede simular datos con probabilidades no
simétricas para los resultados (por ejemplo, un 90\% de probabilidad de éxito)
usando el parámetro \texttt{prob} de \texttt{sample}, como ser:

\begin{lstlisting}[language=R]
> sample(c("succ", "fail"), 10, replace=T, prob=c(0.9, 0.1))
[1] "succ" "succ" "succ" "succ" "succ" "succ" "succ" "succ"
[9] "succ" "succ"
\end{lstlisting}

Sin embargo, esta puede no ser la mejor manera de generar una muestra de este
tipo. Ver la discusión posterior de la distribución binomial.

%%%%%%%%%%%%%%%%%%%%%%%%%%%%%%%%%%%%%%%%%%%%%%%%%%%%%%%%%%%%%%%%%%%%%%%%%%%%%%%
\section{Cálculos de probabilidad y combinatoria}
%%%%%%%%%%%%%%%%%%%%%%%%%%%%%%%%%%%%%%%%%%%%%%%%%%%%%%%%%%%%%%%%%%%%%%%%%%%%%%%

Volvamos al caso del muestreo sin reemplazo, específicamente
\texttt{sample(1:40,5)}. La probabilidad de obtener un número dado como ser, el
primero de la muestra debe ser de $1/40$, la del siguiente es de $1/39$, y así
sucesivamente. La probabilidad entonces  de obtener una determinada combinación
de 5 números,  debe ser de $1/(40 \times 39 \times 38 \times 37 \times 36)$.  En
\textbf{R}, puede utilizar la función \texttt{prod}, para calcular el producto
de un vector de números:

\begin{lstlisting}[language=R]
> 1/prod(40:36)
[1] 1.266449e-08
\end{lstlisting}

Sin embargo, tenga en cuenta que esta es la probabilidad de que se den los
números en un orden determinado. Si se tratara de un juego como el de la
lotería, entonces preferiría estar interesado en la probabilidad de adivinar
correctamente un determinado conjunto de cinco números.  Por lo tanto, también
es necesario incluir los casos que dan los mismos números pero en un orden
diferente.  Puesto que obviamente la probabilidad de cada caso va a ser la
misma, todo lo que tenemos que hacer es averiguar cuántos casos de este tipo hay
y multiplicarlos por eso. Hay cinco posibilidades para el primer número, y para
cada uno de ellos hay cuatro posibilidades para el segundo, y así sucesivamente;
es decir, el número es $5 \times 4 \times 3 \times 2 \times 1$. ¡Este número
también se escribe como $5!$ (factorial de 5)!. Así que la probabilidad de un
cupón ganador de la lotería, sería:

\begin{lstlisting}[language=R]
> prod(5:1)/prod(40:36)
[1] 1.519738e-06
\end{lstlisting}

Hay otra forma de llegar al mismo resultado. Note que como el conjunto real de
números es irrelevante, todos los conjuntos de cinco números deben tener la
misma probabilidad. Así que todo lo que tenemos que hacer es calcular el número
de maneras de elegir 5 números de un total de 40. Esto se puede definir como:

\begin{gather*}
\left(
    \begin{array}{c}
      40\\
      5
  \end{array}
\right) = \frac{40!}{5!35!} = 658008
\end{gather*}

En \textbf{R}, se puede utilizar la función \texttt{choose} para calcular este
número, y por lo tanto la probabilidad es:

\begin{lstlisting}[language=R]
> 1/choose(40,5)
[1] 1.519738e-06
\end{lstlisting}

%%%%%%%%%%%%%%%%%%%%%%%%%%%%%%%%%%%%%%%%%%%%%%%%%%%%%%%%%%%%%%%%%%%%%%%%%%%%%%%
\section{Distribuciones discretas}
%%%%%%%%%%%%%%%%%%%%%%%%%%%%%%%%%%%%%%%%%%%%%%%%%%%%%%%%%%%%%%%%%%%%%%%%%%%%%%%

Al examinar las réplicas independientes de un experimento binario, normalmente
no nos interesaría saber si cada caso es un éxito o un fracaso, sino más bien el
número total de éxitos (o fracasos). Obviamente, este número es aleatorio ya que
depende de la salida aleatoria individual, y por lo tanto se le llama variable
aleatoria. En este caso, es una \textit{variable aleatoria} de valor discreto
que puede tomar valores 0, 1, .... \textit{n}, donde \textit{n} es el número de
réplicas. Las variables aleatorias continuas se verán más tarde.

Una variable aleatoria \textit{X} tiene una distribución de probabilidad que puede describirse utilizando
probabilidades de puntos f ( x ) = P ( X = x ) o la distribución acumulativa
función F ( x ) = P ( X <= x ) . En el caso que nos ocupa, la distribución puede ser
...resultó tener las probabilidades de los puntos.


A random variable X has a probability distribution that can be described using
point probabilities f ( x ) = P ( X = x ) or the cumulative distribution
function F ( x ) = P ( X <= x ) . In the case at hand, the distribution can be
worked out as having the point probabilities.

\section{Continuous distributions}
\section{The built-in distributions in R}
\subsection{Densities}
\subsection{Cumulative distribution functions}
\subsection{Quantiles}
\subsection{Random numbers}
\section{Ejercicios}
\newpage

%%%%%%%%%%%%%%%%%%%%%%%%%%%%%%%%%%%%%%%%%%%%%%%%%%%%%%%%%%%%%%%%%%%%%%%%%%%%%%%
\chapter{Descriptive statistics and graphics}
%%%%%%%%%%%%%%%%%%%%%%%%%%%%%%%%%%%%%%%%%%%%%%%%%%%%%%%%%%%%%%%%%%%%%%%%%%%%%%%
\section{Resumen estadístico para un solo grupo} \label{resestad}
\section{Graphical display of distributions}
\section{Graphics for grouped data}
\section{Tables} \label{tables}
\section{Graphical display of tables}


%%%%%%%%%%%%%%%%%%%%%%%%%%%%%%%%%%%%%%%%%%%%%%%%%%%%%%%%%%%%%%%%%%%%%%%%%%%%%%%
\chapter{Manejo avanzado de datos} \label{datosavanzados}
%%%%%%%%%%%%%%%%%%%%%%%%%%%%%%%%%%%%%%%%%%%%%%%%%%%%%%%%%%%%%%%%%%%%%%%%%%%%%%%
\section{Recoding variables}
\subsection{The cut function}
\subsection{Manipulating factor levels}
\subsection{Working with dates}
\subsection{Recoding multiple variables}
\section{Conditional calculations}
\section{Combining and restructuring data frames}
\subsection{Appending frames} \label{appending frames}
\subsection{Merging data frames}
\subsection{Reshaping data frames}
\section{Per-group and per-case procedures}
\section{Time splitting}
\section{Ejercicios}


\end{document}
