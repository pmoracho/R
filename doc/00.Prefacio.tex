%%%%%%%%%%%%%%%%%%%%%%%%%%%%%%%%%%%%%%%%%%%%%%%%%%%%%%%%%%%%%%%%%%%%%%%%%%%%%%%
%% Prefacio
%%%%%%%%%%%%%%%%%%%%%%%%%%%%%%%%%%%%%%%%%%%%%%%%%%%%%%%%%%%%%%%%%%%%%%%%%%%%%%%
\chapter*{Prefacio}

\textbf{R} es un programa informático para estadístas, disponible a través de
Internet bajo la Licencia Pública General (\textbf{GPL}). Es decir, se
suministra con una licencia que nos permite utilizarlo libremente, distribuirlo
o incluso venderlo, siempre que el receptor tenga los mismos derechos y el
código fuente esté disponible libremente. Existen versiones para Microsoft
Windows XP o posteriores, para una variedad de plataformas Unix y Linux, y para
Apple Macintosh OS X.

\textbf{R} proporciona un entorno en el que puede realizar análisis
estadísticos y producir gráficos.  En realidad es un lenguaje de programación
completo, aunque eso, en este libro sólo se describe marginalmente. Aquí nos
conformamos con aprender los conceptos elementales y ver varios ejemplos.

\textbf{R} está diseñado de tal manera que siempre es posible realizar más
cálculos sobre los resultados de un procedimiento estadístico. Además, la
funcionalidad para la presentación gráfica de los datos permite tanto métodos
simples, como por ejemplo \texttt{plot(x, y)}, como también la posibilidad de
un control mucho más fino del aspecto de las gráficas.

El hecho que \textbf{R} esté basado en un lenguaje formal de computadora le da
una tremenda flexibilidad.  Otros sistemas presentan interfaces más sencillas
en términos de menús y formularios, pero a menudo la aparente facilidad de uso
se convierte en un obstáculo a largo plazo. Aunque las estadísticas elementales
se presentan a menudo como una colección de procedimientos fijos, el análisis
de datos moderadamente complejos requiere la creación de modelos estadísticos
ad-hoc, lo que hace que la flexibilidad añadida de \textbf{R} sea altamente
deseable.

\textbf{R} debe su nombre a una típica humorada de Internet. Es posible que
haya oído hablar del lenguaje de programación C (cuyo nombre es otra historia
en sí misma). Inspirados por esto, Becker y Chambers eligieron, a principios de
los años ochenta, llamar a su nuevo lenguaje de programación estadística S.
Este lenguaje se desarrolló aún más con el producto comercial S-PLUS, que a
finales de la década era de uso generalizado entre los estadísticos de todo
tipo. Ross Ihaka y Robert Gentleman de la Universidad de Auckland, Nueva
Zelanda, eligieron escribir una versión reducida de S para propósitos de
enseñanza, y ¿qué era más natural que elegir la letra inmediatamente anterior?
Las iniciales de Ross y Robert también pueden haber jugado un papel importante
en la elección del nombre.

En 1995, Martin Maechler persuadió a Ross y Robert para que publicaran el
código fuente de \textbf{R} bajo la licencia \textbf{GPL}. Esto coincidió con
el auge del software de código abierto impulsado por el sistema Linux.
\textbf{R} pronto logró llenar un hueco en gente como yo, que tenía la intención
de usar Linux para computación estadística, pero no tenía ningún paquete
estadístico disponible en ese momento. Se creó una lista de correo para la
comunicación de informes de fallos y discusiones sobre el desarrollo de
\textbf{R}.

En agosto de 1997, me invitaron a unirme a un equipo internacional extendido
cuyos miembros colaboran a través de Internet y que ha controlado el desarrollo
de \textbf{R} desde entonces. Posteriormente, el equipo básico se amplió varias
veces y actualmente cuenta con 19 miembros. El 29 de febrero de 2000, se publicó
la versión 1.0.0.0. Al momento de este escrito, la versión actual es la 2.6.2.
Este libro se basó originalmente en una serie de notas elaboradas para el curso
de Estadística Básica para Investigadores en Salud de la Facultad de Ciencias de
la Salud de la Universidad de Copenhague. El curso tenía como objetivo
principal, apuntar a los estudiantes del doctorado en medicina. Sin embargo, el
material se ha revisado sustancialmente y espero que sea útil para un público
más amplio, aunque sigue habiendo algunos sesgos bioestadísticos, especialmente
en la elección de los ejemplos. En los últimos años, el curso de Práctica
Estadística en Epidemiología, que se ha celebrado anualmente en Tartu (Estonia),
ha sido una fuente importante de inspiración y experiencia en la introducción de
jóvenes estadísticos y epidemiólogos a \textbf{R}.  Este libro no es un manual
para \textbf{R}.  La idea es introducir una serie de conceptos y técnicas
básicas que deberían permitir al lector comenzar con estadísticas prácticas. En
cuanto a los métodos prácticos, el libro cubre una currícula razonable para los
estudiantes de primer año de estadística teórica, así como para los estudiantes
de ingeniería. Estos grupos eventualmente necesitarán ir más allá y estudiar
modelos más complejos, así como técnicas generales que involucren programación
real en el lenguaje \textbf{R}.

Para los áreas en las que las estadísticas elementales se enseñan
principalmente como una herramienta, el libro va un poco más allá de lo que se
enseña comúnmente a nivel universitario. Los métodos de regresión múltiple o el
análisis de experimentos multifactoriales rara vez se enseñan a ese nivel, pero
pueden convertirse rápidamente en esenciales para la investigación práctica. He
recopilado los métodos más simples al principio para hacer el libro más legible
también en un nivel elemental. Sin embargo, para mantener el material técnico
consistente, los capítulos 1 y 2 sí incluyen material que algunos lectores
querrán omitir. Por lo tanto, el libro pretende ser útil para varios grupos,
pero no pretenderé que pueda valerse por sí solo para ninguno de ellos. He
incluido breves secciones teóricas en relación con los diversos métodos, pero
más que como material didáctico, éstos deberían servir de recordatorios o
quizás como aperitivos para los lectores que son nuevos en el mundo de la
estadística.

