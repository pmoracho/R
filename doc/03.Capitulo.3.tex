%%%%%%%%%%%%%%%%%%%%%%%%%%%%%%%%%%%%%%%%%%%%%%%%%%%%%%%%%%%%%%%%%%%%%%%%%%%%%%%
\chapter{Probabilidad y distribuciones}
%%%%%%%%%%%%%%%%%%%%%%%%%%%%%%%%%%%%%%%%%%%%%%%%%%%%%%%%%%%%%%%%%%%%%%%%%%%%%%%

Los conceptos de aleatoriedad y probabilidad son fundamentales para las
estadísticas. Es un hecho empírico que la mayoría de los experimentos e
investigaciones no son perfectamente reproducibles. El grado de
irreproducibilidad puede variar: Algunos experimentos en física pueden producir
datos que son exactos hasta muchos decimales, mientras que los datos sobre
sistemas biológicos son típicamente mucho menos confiables. Sin embargo, la
visión de los datos como algo que proviene de una distribución estadística es
vital para entender los métodos estadísticos. En esta sección se esbozan las
ideas básicas de probabilidad y las funciones que tiene \textbf{R} para el
muestreo aleatorio y el manejo de las distribuciones teóricas.

%%%%%%%%%%%%%%%%%%%%%%%%%%%%%%%%%%%%%%%%%%%%%%%%%%%%%%%%%%%%%%%%%%%%%%%%%%%%%%%
\section{Muestreo aleatorio}
%%%%%%%%%%%%%%%%%%%%%%%%%%%%%%%%%%%%%%%%%%%%%%%%%%%%%%%%%%%%%%%%%%%%%%%%%%%%%%%

Gran parte de los primeros trabajos en la teoría de la probabilidad se referían
a juegos y apuestas, basados en consideraciones de simetría. Por lo tanto, la
noción básica de una  muestra aleatoria es la de repartir de una baraja bien
barajada o recoger bolas numeradas de una urna bien revuelta.

En \textbf{R}, se pueden simular estas situaciones con la función
\texttt{sample}. Si quieres elegir cinco números al azar del conjunto de
\texttt{1:40}, entonces se puede escribir:

\begin{lstlisting}[language=R]
> sample(1:40,5)
[1] 36 37 26 24  3
\end{lstlisting}

El primer parámetro (\texttt{x}) es el vector de valores posibles de dónde
obtendremos la muestra y el segundo (\texttt{size}) es el tamaño de la muestra.
En realidad, \texttt{sampe(40,5)} sería suficiente ya que un solo número se
interpreta como la longitud de una secuencia de números enteros desde el 1.

Note que el comportamiento predeterminado de \texttt{sample} es el de
\textit{muestreo sin reemplazo}. Es decir, las muestras no contendrán ningún
número repetido, y \texttt{size} obviamente no puede ser mayor que la longitud
del vector a muestrear. Si quiere muestreo con reemplazo, entonces necesita
agregar el parámetro \texttt{replace=TRUE}.

El muestreo con reemplazo es adecuado para modelar lanzamientos de monedas o
lanzamientos de un dado. Así, por ejemplo, para simular 10 lanzamientos de
monedas podríamos escribir:

\begin{lstlisting}[language=R]
> sample(c("H","T"), 10, replace=T)
[1] "T" "T" "T" "T" "T" "H" "H" "T" "H" "T"
\end{lstlisting}

En un lanzamiento normal de monedas, la probabilidad de cara debe ser igual a la
probabilidad de cruz, pero la idea de un evento aleatorio no se limita a casos
simétricos.  Podría aplicarse igualmente bien a otros casos, como el resultado
exitoso de un procedimiento quirúrgico. Esperemos que haya más de un 50\% de
posibilidades de que esto suceda. Puede simular datos con probabilidades no
simétricas para los resultados (por ejemplo, un 90\% de probabilidad de éxito)
usando el parámetro \texttt{prob} de \texttt{sample}, como ser:

\begin{lstlisting}[language=R]
> sample(c("succ", "fail"), 10, replace=T, prob=c(0.9, 0.1))
[1] "succ" "succ" "succ" "succ" "succ" "succ" "succ" "succ"
[9] "succ" "succ"
\end{lstlisting}

Sin embargo, esta puede no ser la mejor manera de generar una muestra de este
tipo. Ver la discusión posterior de la distribución binomial.

%%%%%%%%%%%%%%%%%%%%%%%%%%%%%%%%%%%%%%%%%%%%%%%%%%%%%%%%%%%%%%%%%%%%%%%%%%%%%%%
\section{Cálculos de probabilidad y combinatoria}
%%%%%%%%%%%%%%%%%%%%%%%%%%%%%%%%%%%%%%%%%%%%%%%%%%%%%%%%%%%%%%%%%%%%%%%%%%%%%%%

Volvamos al caso del muestreo sin reemplazo, específicamente
\texttt{sample(1:40,5)}. La probabilidad de obtener un número dado como ser, el
primero de la muestra debe ser de $1/40$, la del siguiente es de $1/39$, y así
sucesivamente. La probabilidad entonces  de obtener una determinada combinación
de 5 números,  debe ser de $1/(40 \times 39 \times 38 \times 37 \times 36)$.  En
\textbf{R}, puede utilizar la función \texttt{prod}, para calcular el producto
de un vector de números:

\begin{lstlisting}[language=R]
> 1/prod(40:36)
[1] 1.266449e-08
\end{lstlisting}

Sin embargo, tenga en cuenta que esta es la probabilidad de que se den los
números en un orden determinado. Si se tratara de un juego como el de la
lotería, entonces preferiría estar interesado en la probabilidad de adivinar
correctamente un determinado conjunto de cinco números.  Por lo tanto, también
es necesario incluir los casos que dan los mismos números pero en un orden
diferente.  Puesto que obviamente la probabilidad de cada caso va a ser la
misma, todo lo que tenemos que hacer es averiguar cuántos casos de este tipo hay
y multiplicarlos por eso. Hay cinco posibilidades para el primer número, y para
cada uno de ellos hay cuatro posibilidades para el segundo, y así sucesivamente;
es decir, el número es $5 \times 4 \times 3 \times 2 \times 1$. ¡Este número
también se escribe como $5!$ (factorial de 5)!. Así que la probabilidad de un
cupón ganador de la lotería, sería:

\begin{lstlisting}[language=R]
> prod(5:1)/prod(40:36)
[1] 1.519738e-06
\end{lstlisting}

Hay otra forma de llegar al mismo resultado. Note que como el conjunto real de
números es irrelevante, todos los conjuntos de cinco números deben tener la
misma probabilidad. Así que todo lo que tenemos que hacer es calcular el número
de maneras de elegir 5 números de un total de 40. Esto se puede definir como:

\begin{gather*}
\left(
    \begin{array}{c}
      40\\
      5
  \end{array}
\right) = \frac{40!}{5!35!} = 658008
\end{gather*}

En \textbf{R}, se puede utilizar la función \texttt{choose} para calcular este
número, y por lo tanto la probabilidad es:

\begin{lstlisting}[language=R]
> 1/choose(40,5)
[1] 1.519738e-06
\end{lstlisting}

%%%%%%%%%%%%%%%%%%%%%%%%%%%%%%%%%%%%%%%%%%%%%%%%%%%%%%%%%%%%%%%%%%%%%%%%%%%%%%%
\section{Distribuciones discretas}
%%%%%%%%%%%%%%%%%%%%%%%%%%%%%%%%%%%%%%%%%%%%%%%%%%%%%%%%%%%%%%%%%%%%%%%%%%%%%%%

Al examinar las réplicas independientes de un experimento binario, normalmente
no nos interesaría saber si cada caso es un éxito o un fracaso, sino más bien el
número total de éxitos (o fracasos). Obviamente, este número es aleatorio ya que
depende de la salida aleatoria individual, y por lo tanto se le llama variable
aleatoria. En este caso, es una \textit{variable aleatoria} de valor discreto
que puede tomar valores 0, 1, .... \textit{n}, donde \textit{n} es el número de
réplicas. Las variables aleatorias continuas se verán más tarde.

Una variable aleatoria \textit{X} tiene una distribución de probabilidad que puede describirse utilizando
probabilidades de puntos f ( x ) = P ( X = x ) o la distribución acumulativa
función F ( x ) = P ( X <= x ) . En el caso que nos ocupa, la distribución puede ser
...resultó tener las probabilidades de los puntos.


A random variable X has a probability distribution that can be described using
point probabilities f ( x ) = P ( X = x ) or the cumulative distribution
function F ( x ) = P ( X <= x ) . In the case at hand, the distribution can be
worked out as having the point probabilities.

\section{Continuous distributions}
\section{The built-in distributions in R}
\subsection{Densities}
\subsection{Cumulative distribution functions}
\subsection{Quantiles}
\subsection{Random numbers}
\section{Ejercicios}
\newpage
