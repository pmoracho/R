%%%%%%%%%%%%%%%%%%%%%%%%%%%%%%%%%%%%%%%%%%%%%%%%%%%%%%%%%%%%%%%%%%%%%%%%%%%%%%%
\chapter{Estadística descriptiva y gráficos}
%%%%%%%%%%%%%%%%%%%%%%%%%%%%%%%%%%%%%%%%%%%%%%%%%%%%%%%%%%%%%%%%%%%%%%%%%%%%%%%

Antes de entrar en el análisis y modelado estadístico real de un conjunto de
datos, a menudo es útil hacer algunas caracterizaciones simples de los datos en
términos de estadísticas y gráficos resumidos.

%%%%%%%%%%%%%%%%%%%%%%%%%%%%%%%%%%%%%%%%%%%%%%%%%%%%%%%%%%%%%%%%%%%%%%%%%%%%%%%
\section{Resumen estadístico para un grupo} \label{resestad}
%%%%%%%%%%%%%%%%%%%%%%%%%%%%%%%%%%%%%%%%%%%%%%%%%%%%%%%%%%%%%%%%%%%%%%%%%%%%%%%

Es fácil calcular estadísticas de resumen simples con \textbf{R}. A continuación se
explica cómo calcular la media, la desviación estándar, la varianza y la
mediana.

\begin{lstlisting}[language=R]
    > x <- rnorm(50)
    > mean(x)
    [1] 0.1896194
    > sd(x)
    [1] 1.089223
    > var(x)
    [1] 1.186406
    > median(x)
    [1] 0.3191134
\end{lstlisting}

Observe que el ejemplo comienza con la generación de un vector \texttt{x} con 50
observaciones distribuidas normalmente. Se utiliza en varios ejemplos a lo largo
de esta sección. Al reproducir los ejemplos, no va a obtener exactamente los
mismos resultados ya que sus números aleatorios serán diferentes.

Los cuantiles empíricos se pueden obtener con la función \texttt{quantile} de
esta manera:

\begin{lstlisting}[language=R]
    > quantile(x)
    0%        25%        50%        75%       100%
    -2.4454931 -0.4995763  0.3191134  0.8295342  2.3352865
\end{lstlisting}

Como puede se puede observar, por defecto obtenemos el mínimo, el máximo y los
tres cuartiles: los \textit{cuartiles} 0.25, 0.50 y 0.75 llamados así porque
corresponden a un división en cuatro partes. Del mismo modo, tenemos los
\textit{deciles} de 0,1, 0,2,. . . , 0,9, \textit{centiles} o
\textit{percentiles}. La diferencia entre el primer y tercer cuartil es llamado
rango intercuartílico (IQR) y a veces se utiliza como una alternativa sólida a
la desviación estándar.

También es posible obtener otros cuantiles; esto se hace agregando un argumento
que contiene los puntos porcentuales deseados. Esto, por ejemplo, es cómo
hacemos para obtener los deciles:


\begin{lstlisting}[language=R]
    > pvec
    [1] 0.0 0.1 0.2 0.3 0.4 0.5 0.6 0.7 0.8 0.9 1.0
    > quantile(x, pvec)
    0%        10%        20%        30%        40%        50%        60%
    -2.4454931 -1.3395813 -0.7258860 -0.4121909  0.0182647  0.3191134  0.6880762
    70%        80%        90%       100%
    0.7846998  0.9057485  1.6282719  2.3352865
\end{lstlisting}

Tenga en cuenta que hay varias definiciones posibles de cuantiles empíricos.
The one R uses by default is based on a sum polygon where the ith ranking
observation is the ( i − 1 ) / ( n − 1 ) quantile and intermediate quantiles are
obtained by linear interpolation. It sometimes confuses students that in
a sample of 10 there will be 3 observations below the first quartile with
this definition. Other definitions are available via the type argument to
quantile.

\section{Exhibición gráfica de distribuciones}



\subsection{Histograms}
\subsection{Empirical cumulative distribution}
\subsection{Q–Q plots}\label{qqplots}
\subsection{Boxplots}
\section{Summary statistics by groups}
\section{Graphics for grouped data}
\subsection{Histograms}
\subsection{Parallel boxplots}
\subsection{Stripcharts}
\section{Tables} \label{tables}
\subsection{Generating tables}
\subsection{Marginal tables and relative frequency}
\section{Graphical display of tables}
\subsection{Barplots}
\subsection{Dotcharts}
\subsection{Piecharts}
\section{Exercises}

